\documentclass{article}

\usepackage{amsmath}
\usepackage{amssymb}
\usepackage{amsthm}
\usepackage{enumerate}
\usepackage{bbm}
\usepackage{lipsum}
\usepackage{fancyhdr}
\usepackage{calrsfs}
\usepackage{tikz-cd} 

\newtheorem{theorem}{Theorem}[section] 
\newtheorem{proposition}{Proposition}[section] 
\newtheorem{definition}{Definition}[section] 
\newtheorem{lemma}{Lemma}[section] 
\newtheorem{notation}{Notation}[section] 
\newtheorem{remark}{Remark}[section] 
\newtheorem{corollary}{Corollary}[section] 
\newtheorem{terminology}{Terminology}[section] 
\newtheorem{example}{Example}[section] 

\DeclareMathOperator{\diam}{diam}
\DeclareMathOperator{\rk}{rk}
\DeclareMathOperator{\rank}{rank}
\DeclareMathOperator{\Hom}{Hom}
\DeclareMathOperator{\Dom}{Dom}
\DeclareMathOperator{\grad}{grad}
\DeclareMathOperator{\Span}{Span}
\DeclareMathOperator{\interior}{int}
\DeclareMathOperator{\ind}{ind}
\DeclareMathOperator{\supp}{supp}
\DeclareMathOperator{\sgn}{sgn}
\DeclareMathOperator{\ob}{ob}
\DeclareMathOperator{\Spec}{Spec}
\DeclareMathOperator{\PreSh}{PreSh}
\DeclareMathOperator{\Fun}{Fun}
\DeclareMathOperator{\Ker}{Ker}
\DeclareMathOperator{\Image}{Im}
\DeclareMathOperator{\Ad}{Ad}
\DeclareMathOperator{\ad}{ad}
\DeclareMathOperator{\End}{End}
\DeclareMathOperator{\GL}{GL}
\DeclareMathOperator{\SL}{SL}
\DeclareMathOperator{\Lie}{Lie}


\title{V4A1 Sheet 5}
\author{So Murata}
\date{2024/2025 Winter Semester - Uni Bonn}

\begin{document}
\maketitle
Suppose the statement is false, then there exists $u_k$ for each $k\in\mathbb{N}$ such that
\begin{equation*}
[D^2u_k]_{C^{0,\alpha}}>k[\Delta u_k]_{C^{0,\alpha}}.
\end{equation*}
We can assume $[D^2u_k]_{C^{0,\alpha}}=1$ by dividing $u$ by $[D^2u_k]_{C^{0,\alpha}}$. Thus we have the inequality
\begin{equation*}
{\frac 1 k}>[\Delta u_k]_{C^{0,\alpha}}.
\end{equation*}
Using the pigeon hole principle on the definition of supremums, we find $i,j,k=1,\cdots,n$ such that infinitely many $x_l\in\mathbb{R}^n$ and $h_l>0$ we have
\begin{equation*}
{\frac {|D^2_{ij}u_l(x_l+h_l\underline{e}_k)-D^2_{ij}u_l(x_l)|} {h_l^\alpha}}\geq {\frac 1 {2n^3}}.
\end{equation*}

We define a shifted $u_l$ corresponding those $x_l$ and $h_l$ by 
\begin{equation*}
\tilde{u}_l(x) = h_l^{-2-\alpha}u_l(x_l+h_lx).
\end{equation*}
To summarize, we have
\begin{equation*}
[D^2\tilde{u}_l]_{C^{0,\alpha}}=1,\quad [\Delta \tilde{u}_l]_{C^{0,\alpha}}<{\frac 1 l},\quad |D^2_{ij}\tilde{u}_l(x)-D^2_{ij}\tilde{u}_l(0)|\geq {\frac 1 {2n^3}}.
\end{equation*}
By adding appropriate second order polynomial, we conclude that 
\begin{equation*}
\tilde{u}_l(0)=D\tilde{u}_l(0)=D^2\tilde{u}_l(0)=0.
\end{equation*}
Also we notice that at $\underline{e}_k$ we have
\begin{equation*}
\tilde{u}_l(\underline{e}_k) \geq {\frac 1 {2n^3}}.
\end{equation*}
By the definition of $[\cdot]_{C^{0,\alpha}}$, it is clear that $\tilde{u}_l$ is bounded. Transforming the definition, we condlue
\begin{equation*}
|\tilde{u}_l(x)|\leq C|h_lx|^{2+\alpha},\quad |D\tilde{u}_l(x)|\leq C|h_lx|^{1+\alpha},\quad |D^2\tilde{u}_l(x)|\leq |h_lx|^{\alpha}.
\end{equation*}
The sequence is globally bounded and equicontinuous by the definition of Holder condition and the assumption on $\alpha$. Furthermore, by appropriately large enough $R$, and ArzelàAscoli theorem on $B_R(0)$, the definition of $[\cdot]_{C^{0,\alpha}}$ on $\Delta u_l$, we conclude that 
\begin{equation*}
\Delta_u\equiv 0, \quad |D^2\tilde{u}(x)|\leq |h_lx|^{\alpha},\quad D^2_{ij}u_(\underline{e}_k)\geq  {\frac 1 {2n^3}}.
\end{equation*}

\begin{equation*}
\lim_{|x|\to\infty} {\frac {|u(x)|} {|x|^3}} =0.
\end{equation*}
Thus, we conclude that $u$ is a polynomial of at most degree 2. Therefore but at $0$ all the derivatives vanish, we conclude $u\equiv 0$. This is a contradiction.
\end{document}