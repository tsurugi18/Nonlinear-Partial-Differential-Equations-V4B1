\documentclass{article}
\usepackage[utf8]{inputenc}
\usepackage{amsmath,amsfonts,amsthm,amssymb}
\usepackage{graphicx}
\usepackage{lmodern}
\usepackage[T1]{fontenc}
\usepackage{mathtools}
\usepackage{tikz-cd}
\usepackage[a4paper, total={6in, 8in}]{geometry}
\usepackage{xfrac}
\usepackage{enumitem}

\newcommand{\R}{\mathbb{R}}
\newcommand{\Z}{\mathbb{Z}}
\newcommand{\N}{\mathbb{N}}
\newcommand{\Q}{\mathbb{Q}}
\newcommand{\C}{\mathbb{C}}
\newcommand{\F}{\mathbb{F}}
\newcommand{\abs}[1]{\vert#1\vert}
\newcommand{\systeme}[2]{\begin{equation}
    \left\{
    \begin{aligned}
        &#1\\
        &#2\\
    \end{aligned}
    \right.
\end{equation}}
\newcommand{\limit}[2]{\lim_{#1\to#2}}
\newcommand{\power}[2]{\left(#1\right)^{#2}}
%L'accent aigu (é)
%L'accent grave (à, è, ù)
%L'accent circonflexe or "chapeau" (â, ê, î, ô, û)
%La cédille (ç)
%Le tréma (ë, ï, ü, ö)
\newtheorem{theorem}{Theorem}
\newtheorem{definition}{Definition}
\newtheorem{lemma}{Lemma}
\newtheorem{proposition}{Proposition}
\newtheorem{corollary}{Corollary}
\newtheorem{remark}{Remark}
\newtheorem{exercise}{Exercise}
\newtheorem{example}{Example}
\DeclareMathOperator{\Aut}{\textup{Aut}}
\DeclareMathOperator{\Frac}{\textup{Frac}}
\DeclareMathOperator{\ev}{\textup{ev}}
\DeclareMathOperator{\GL}{\textup{GL}}
\DeclareMathOperator{\PGL}{\textup{PGL}}
\DeclareMathOperator{\End}{\textup{End}}
\DeclareMathOperator{\SO}{\textup{SO}}
\DeclareMathOperator{\Sym}{\textup{Sym}}
\DeclareMathOperator{\tr}{\textup{tr}}
\DeclareMathOperator{\diag}{\textup{diag}}
\DeclareMathOperator{\diam}{\textup{diam}}
\DeclareMathOperator{\spt}{\textup{spt}}
\DeclareMathOperator{\loc}{\textup{loc}}
\DeclareMathOperator{\Lip}{\textup{Lip}}
\DeclareMathOperator{\Div}{\textup{div}}
\DeclareMathOperator*{\argmin}{arg\,min}
\DeclareMathOperator*{\argmax}{arg\,max}

\title{Sheet 2 Solutions}
\author{Toni Abi Aoun, So Murata}
\date{}

\begin{document}
\maketitle

\section*{Exercise 2.1}

\subsection*{2.1.i}

For a smooth function $u\in C^\infty(\mathbb{R}^n)$ which also belongs to $H^k(\mathbb{R}^n)$, we derive
\begin{equation*}
\widehat{D^\alpha u} = (iy)^\alpha \hat{u}.
\end{equation*}

Since $D^\alpha u\in L^2(\mathbb{R}^n)$, we derive the right hand side $(iy)^\alpha \hat{u}$ is also in $L^2(\mathbb{R}^n)$ for each $\alpha$. by choosing $\alpha = (k,0,\cdots,0),(0,k,\cdots,0),\cdots,(0,0,\cdots,k)$, we derive
\begin{equation*}
\int_{\mathbb{R}^n}\vert y\vert^{2k}\vert\hat{u}\vert^2 dy \leq \Vert D^k u\Vert^2_{L^2(\mathbb{R}^n)}.
\end{equation*}
Since Fourier transform is an isometry we derive and there are other norms of derivatives added,
\begin{equation*}
\int_{\mathbb{R}^n}1+\vert y\vert^{2k}\vert\hat{u}\vert^2dx\leq \Vert u\Vert^2_{H^k}.
\end{equation*}
We have for $a,b>0$ and $s>0$, $(a+b)^s\leq 2^s(a^s+b^s)$.
We conclude that $\Vert (1+\vert y\vert^{k})\hat{u}\Vert^2\leq\Vert u\Vert^2_{H^k}$ thus $ (1+\vert y\vert^{k})\hat{u}$is in $L^2(\mathbb{R}^n)$. \\
\par On the other hand $(1+\vert y\vert^{k})\hat{u}\in L^2(\mathbb{R}^n)$ then for $|\alpha|\leq k$, we have 
\begin{equation}
\label{L2_derivative}
\Vert (iy)^\alpha\hat{u}\Vert_{L^2} \leq \int_{\mathbb{R}^n}\Vert y\Vert^{2|\alpha|}|\hat{u}|^2 dy\leq C\Vert(1+\vert y\vert^{k})^2\hat{u}\Vert_{L^2(\mathbb{R}^n)}.
\end{equation}
Let us denote $u_\alpha = {\frac 1 {2\pi}}\int_{\mathbb{R}^n}((iy)^\alpha\hat{u})e^{iyx}dy$ be the image of the inverse fourier transform of $((iy)^\alpha\hat{u})$. Then

\begin{equation*}
\int_{\mathbb{R}^n}(D^\alpha\varphi)\overline{u}dx=\int_{\mathbb{R}^n}\widehat{D^\alpha\varphi}\overline{\hat{u}}dx = \int_{\mathbb{R}^n}(iy)^\alpha\hat{\varphi}\overline{\hat{u}}dx =(-1)^{|\alpha|} \int_{\mathbb{R}^n}\varphi u_\alpha dx.
\end{equation*}

By Equation \ref{L2_derivative}, $u_\alpha$ is in $L^2$, therefore this is a weak derivative of $u$ and $u$ is in $H^k$.
\section*{Exercise 2.2}

\begin{equation*}
Lu = -\sum_{i=1}^n{\frac {\partial^2 u} {\partial x_i^2}}.
\end{equation*}

We let $B_0[u,v] = -\sum_{i=1}^n \partial_{x_i}u\partial_{x_i}v.$ 

\subsection{Exercise 2.2.i}

With Poincare inequality, we see that
\begin{equation*}
\Vert u\Vert _{H^1_0(\mathbb{R})^n}\leq C\Vert Du\Vert_{H^1_0(\mathbb{R})^n} \Rightarrow {\frac 1 {C^2}}\Vert u\Vert_{H^1_0(\mathbb{R})^n}^2 \leq \Vert Du\Vert_{H^1_0(\mathbb{R})^n}^2=B_0[u,u].
\end{equation*}
also
\begin{equation*}
\left|B_0[u,v]\right| \leq \Vert u\Vert_{H^1_0(\mathbb{R})^n}\Vert v\Vert_{H^1_0(\mathbb{R})^n}
\end{equation*}
follows from Cauchy-Schwarz inequality. Thus $\gamma=0$ for the existence of weak solutions by Lax-Milgram. We derived from $c>0$ that the equation stated in the sheet has a solution in $H^1_0(\mathbb{R}^n)$ thus in $H^1(\mathbb{R}^n)$.

\subsection*{2.2.ii}

By taking the fourier transform of the equation we get
\begin{equation*}
\sum_{i=1}^n y_i^2\hat{u}+c\hat{u} = \hat{f}.
\end{equation*}
Since $u\in L^2(\mathbb{R}^2)$ so $(1-c)u$ is in $L^2(\mathbb{R}^n)$. We conclude that $(1+|y|^2)\hat{u}$ is in $L^2(\mathbb{R}^n)$. Therefore, $u$ is in $H^2(\mathbb{R}^n)$. 
\subsection*{2.2.iii}
Suppose the statement is true for $k-1$.\\
\par By assumption we have that $(1+|y|^k)\hat{f}\in L^2(\mathbb{R}^n)$. This means that 
\begin{equation*}
(1+|y|^k)(c+|y|^2)\hat{u}\in L^2(\mathbb{R}^n).
\end{equation*}
By the induction hypothesis,
$|y|^k\hat{u}, |y|^2\hat{u}$ are both in $L^2(\mathbb{R}^n)$. Thus we conclude that
$(1+|y|^{k+2})\hat{u}$ is in $L^2(\mathbb{R}^n)$, which completes the claim.

\textbf{Exercise 2.3}\\
Let $\Omega\subset\R^n$ be a bounded open domain with smooth boundary. We denote the dual space of $H_0^1(\Omega)$ by $H^{-1}(\Omega)$. Recall that the standard norm on $H^{-1}(\Omega)$ is given by
\[
\lVert f\rVert_{H^{-1}}:=\sup\left\{\langle f,u\rangle:u\in H_0^1(\Omega),\lVert u\rVert_{H_0^1}\leq 1\right\}.
\]
(i) Show that $L^2(\Omega)$ is continuously embedded in $H^{-1}(\Omega)$ by means of the following identification: for any $v\in L^2(\Omega)$ we can define the functional $\hat{v}:H_0^1(\Omega)\to\R$ via
\[
\hat{v}:u\mapsto\int_{\Omega}uvdx.
\]
\begin{proof}
    For any $v\in L^2(\Omega)$, it is clear that $\hat{v}$ is well defined (using Cauchy-Schwarz) and linear. To prove that $\hat{v}\in H^{-1}(\Omega)$ is remains to prove that it is bounded. For this, take $u\in H_0^1(\Omega)$ then
    \[
    |\hat{v}(u)|=\left|\int_{\Omega}uvdx\right|\leq\lVert v\rVert_{L^2}\lVert u\rVert_{L^2}\leq\lVert v\rVert_{L^2}\lVert u\rVert_{H_0^1}.
    \]
    This proves that $\hat{v}$ is a bounded linear operator (therefore continuous) with operator norm
    \[
    \lVert \hat{v}\rVert\leq\lVert v\rVert_{L^2}.
    \]
\end{proof}
(ii) Let $F\in H^{-1}(\Omega)$. Show that there are $f_0,f_1,\dots,f_n\in L^2(\Omega)$ such that 
\begin{equation}\label{eq}
\langle F,u\rangle=\int_{\Omega}\left(f_0u+\sum_{i=1}^nf_i\frac{\partial u}{\partial x_i}\right)dx
\end{equation}
for all $u\in H_0^1(\Omega)$.
\begin{proof}
    We know that $H^1(\Omega)$ is a Hilbert space with inner product
    \[
    (u,v)=\int_{\Omega}uv+Du\cdot Dvdx\quad\forall u,v\in H^1(\Omega).
    \]
    By definition, $H_0^1(\Omega)$ is the closure of $C_c^\infty(\Omega)$ in $H^1(\Omega)$, in particular it is closed. This implies that $H_0^1(\Omega)$ equipped with the same scalar product is also a Hilbert space. Given $F\in H^{-1}(\Omega)$ a continuous linear functional on $H_0^1(\Omega)$, we can apply Riesz's representation theorem to obtain $f\in H_0^1(\Omega)$ such that
    \[
    \langle F,u\rangle=(f,u)\quad\forall u\in H_0^1(\Omega).
    \]
    Set $f_0=f$ and $f_i=\partial_{x_i}f$ for $i=1,\dots,n$ then $f_0,f_1,\dots,f_n\in L^2(\Omega)$ and we have
    \[
    \langle F,u\rangle=(f,u)=\int_{\Omega}fu+Df\cdot Dudx=\int_{\Omega}f_0u+\sum_{i=1}^nf_i\frac{\partial u}{\partial x_i}dx\quad\forall u\in H_0^1(\Omega).
    \]
\end{proof}
(iii) Show that for $F\in H^{-1}(\Omega)$
\[
\lVert F\rVert_{H^{-1}}=\inf\left\{\left(\sum_{i=0}^n\int_{\Omega}|f_i(x)|^2dx\right)^{\frac{1}{2}}:f_0,f_1,\dots,f_n\in L^2(\Omega)\text{ satisfy }\eqref{eq}\right\}
\]
\begin{proof}
    If $f_0,f_1,\dots,f_n\in L^2(\Omega)$ satisfy \eqref{eq} then for any $u\in H_0^1(\Omega)$ with $\lVert u\rVert_{L^2}\leq 1$, we have
    \[
    \begin{aligned}
    |\langle F,u\rangle|&=\left|\int_{\Omega}f_0u+\sum_{i=1}^nf_i\frac{\partial u}{\partial x_i}dx\right|\\
    &=\int_{\Omega}\left|f_0u+\sum_{i=1}^nf_i\frac{\partial u}{\partial x_i}\right|dx\\
    &\leq\int_{\Omega}\left(f_0^2+\sum_{i=1}^nf_i^2\right)^{\frac{1}{2}}\left(u^2+\sum_{i=1}^n\left|\frac{\partial u}{\partial x_i}\right|^2\right)^\frac{1}{2}dx\\
    &\leq\left(\int_{\Omega}f_0^2+\sum_{i=1}^nf_i^2dx\right)^{\frac{1}{2}}\left(\int_\Omega u^2+\sum_{i=1}^n\left|\frac{\partial u}{\partial x_i}\right|^2dx\right)^{\frac{1}{2}}\\
    &=\left(\sum_{i=1}^n\int_\Omega |f_i(x)|^2dx\right)^\frac{1}{2}\lVert u\rVert_{H_0^1}\\
    &\leq\left(\sum_{i=0}^n|f_i(x)|^2dx\right)^\frac{1}{2}.
\end{aligned}
    \]
    This gives the following bound for the norm of $F$
    \[
    \lVert F\rVert_{H^{-1}}\leq m.
    \]
where $m$ denotes the infimum in the statement.\\
To get the converse inequality, let $f_0,f_1,\dots,f_n$ be given by Riesz's representation theorem as in the previous question, then by taking $u=f$, we get
\[
|\langle F,u\rangle|=|(f,f)|=\lVert f\rVert_{H_0^1}\lVert u\rVert_{H_0^1}.
\]
In the case $F=0$, by the uniqueness in Riesz's representation theorem, we would have $f=0$. If not, we can renormalize so that $f/\lVert f\rVert_{H_0^1}$ is of norm $1$ from which we deduce 
\[
\lVert F\rVert_{H^{-1}}\geq m.
\]
\end{proof}
(iv) Let $n=1$, $\Omega=(-1,1)$. Show that the Delta-distribution $\delta_0$ lies in $H^{-1}(\Omega)$ and find a representation as in \eqref{eq}. Recall that
\[
\langle \delta_0,u\rangle=u(0).
\]
\begin{proof}
Clearly, this operator is linear and for any $u\in H_0^1(\Omega)$, we have
\[
|u(0)|=\left|\int_{-1}^0u'(x)dx\right|\leq\left(\int_{-1}^0|u'(x)|^2dx\right)^\frac{1}{2}\leq\lVert u\rVert_{H_0^1}
\]
which proves that $\delta_0\in H^{-1}(\Omega)$.\\
To obtain a representation as in \eqref{eq}, notice that
\[
u(0)=\int_{-1}^0u'(x)dx=\int_{-1}^1 f_0(x)u(x)+f_1(x)u'(x)dx
\]
where $f_0=0$ and $f_1=1_{(0,1)}$. Since $f_0,f_1\in L^2(\Omega)$, this gives us the desired representation.
\end{proof}
\textbf{Exercise 2.4}\\
Consider the half space $\mathbb{H}^n=\{x\in\R^n:x_n>0\}$ and let $p\in[1,\infty)$.\\
(i) Show that for some constant $C>0$ we have
\[
\int_{R^{n-1}}|u(x_1,\dots,x_{n-1},0)|^pdx_1\dots dx_{n-1}\leq C\lVert u\rVert_{W^{1,p}(\mathbb{H}^n)}^p 
\]
for all $u\in W^{1,p}(\mathbb{H}^n)\cap C^1(\overline{\mathbb{H}^n})$.
\begin{proof}
    Using 
    \[
    \begin{aligned}
    u(x_1,\dots,x_{n-1},0)&=-\int_0^\infty \partial_{x_n}\left(e^{-x_n}u(x_1,\dots,x_{n-1},x_n)\right)dx_n\\
    &=-\int_0^\infty e^{-x_n}\partial_{x_n}u(x_1,\dots,x_n)-e^{-x_n}u(x_1,\dots,x_n)dx_n.
    \end{aligned}
    \]
    Using Jensen's inequality for the probability measure $e^{-x_n}dx_n$ on $(0,\infty)$, we get
    \[
    \begin{aligned}
    |u(x_1,\dots,x_{n-1},0)|^p&=\left|\int_{0}^\infty (\partial_{x_n}u(x_1,\dots,x_n)-u(x_1,\dots,x_n))e^{-x_n}dx_n\right|^p\\
    &\leq\int_0^\infty |\partial_{x_n}u(x_1,\dots,x_n)-u(x_1,\dots,x_n)|^pe^{-x_n}dx_n\\
    &\leq\int_0^\infty 2^{p-1}(|u(x_1,\dots,x_n)|^p+|\partial_{x_n}u(x_1,\dots,x_n)|^p)dx_n,
    \end{aligned}
    \]
    where we used $e^{-x}\leq 1$ for $x>0$ and $(a+b)^p\leq 2^{p-1}(a^p+b^p)$ for $a,b\geq 0$ by convexity of $x\mapsto x^p$ on $(0,\infty)$.\\
    From this we get
    \[
    \begin{aligned}
        \int_{\R^{n-1}}|u(x_1,\dots,x_{n-1},0)|^pdx_1\dots dx_{n-1}&\leq 2^{p-1}\int_{\R^{n-1}}\int_0^\infty |u(x_1,\dots,x_n)|^p+|\partial_{x_n}u(x_1,\dots,x_n)|^pdx_ndx_1\dots dx_{n-1}\\
        &=2^{p-1}\int_{\mathbb{H}^n}|u(x_1,\dots,x_n)|^p+|\partial_{x_n} u(x_1,\dots,x_n)|^pdx_1\dots dx_n\\
        &=2^{p-1}\lVert u\rVert_{W^{1,p}(\mathbb{H}^n)}^p
    \end{aligned}
    \]
\end{proof}
(ii) Show that there is a linear bounded map
\[
T:W^{1,p}(\mathbb{H}^n)\to L^p(\partial\mathbb{H}^n)
\]
such that $Tu=u|_{\partial\mathbb{H}^n}$ for $u\in C(\overline{\mathbb{H}^n})\cap W^{1,p}(\mathbb{H}^n)$.
\begin{proof}
    For every $u\in W^{1,p}(\mathbb{H}^n)\cap C^1(\overline{\mathbb{H}^n})$, define $Tu=u|_{\partial\mathbb{H}^n}$. This map is clearly linear, and by the previous question, it is bounded as a map $W^{1,p}(\mathbb{H}^n)\cap C^1(\overline{\mathbb{H}^n})\to L^p(\partial\mathbb{H}^n)$ for the $W^{1,p}(\mathbb{H}^n)$ norm. Since $W^{1,p}(\mathbb{H}^n)\cap C^1(\overline{\mathbb{H}^n})$ is dense in $W^{1,p}(\mathbb{H}^n)$, the linear operator $T$ extends uniquely to a bounded linear operator $W^{1,p}(\mathbb{H}^n)\to L^p(\partial\mathbb{H}^n)$. To prove that $Tu=u|_{\partial\mathbb{H}^n}$ for all $u\in W^{1,p}(\mathbb{H}^n)\cap C(\overline{\mathbb{H}^n})$, note that for any $u\in W^{1,p}(\mathbb{H}^n)\cap C(\overline{
    \mathbb{H}^n})$, we can always find a sequence $(u_k)\subset W^{1,p}(\mathbb{H}^n)\cap C^1(\overline{\mathbb{H}^n})$ such that $u_k\to u$ in $W^{1,p}(\mathbb{H}^n)$ and $u_k\to u$ pointwise (by taking convolution with good kernels for example). In particular we have $\lim u_k|_{\partial\mathbb{H}^n}=u|_{\partial\mathbb{H}^n}$ pointwise and in $L^p(\partial\mathbb{H}^n)$, so we get
    \[
    Tu=\lim_{k\to\infty}Tu_k=\lim_{k\to\infty}u_k|_{\partial\mathbb{H}^n}=u|_{\partial\mathbb{H}^n}.
    \]
\end{proof}


\end{document}
