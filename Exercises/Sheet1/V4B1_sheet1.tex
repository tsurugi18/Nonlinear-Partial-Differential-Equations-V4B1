\documentclass{article}

\usepackage{amsmath}
\usepackage{amssymb}
\usepackage{amsthm}
\usepackage{enumerate}
\usepackage{bbm}
\usepackage{lipsum}
\usepackage{fancyhdr}
\usepackage{calrsfs}
\usepackage{tikz-cd} 

\newtheorem{theorem}{Theorem}[section] 
\newtheorem{proposition}{Proposition}[section] 
\newtheorem{definition}{Definition}[section] 
\newtheorem{lemma}{Lemma}[section] 
\newtheorem{notation}{Notation}[section] 
\newtheorem{remark}{Remark}[section] 
\newtheorem{corollary}{Corollary}[section] 
\newtheorem{terminology}{Terminology}[section] 
\newtheorem{example}{Example}[section] 

\DeclareMathOperator{\diam}{diam}
\DeclareMathOperator{\Ker}{Ker}
\DeclareMathOperator{\rank}{rank}
\DeclareMathOperator{\Hom}{Hom}
\DeclareMathOperator{\Dom}{Dom}
\DeclareMathOperator{\grad}{grad}
\DeclareMathOperator{\Span}{Span}
\DeclareMathOperator{\interior}{int}
\DeclareMathOperator{\ind}{ind}
\DeclareMathOperator{\supp}{supp}
\DeclareMathOperator{\ob}{ob}
\DeclareMathOperator{\Spec}{Spec}
\DeclareMathOperator{\PreSh}{PreSh}
\DeclareMathOperator{\Fun}{Fun}


\title{Nonlinear Partial Differential Equations Exercise Sheet 1 Solutions}
\author{So Murata}
\date{2024/2025 Winter Semester - Uni Bonn}

\begin{document}
\maketitle
\section*{Exercise 1}

Suppose the inequality does not hold. Then for each $n$, there is $v_n$ such that
\begin{equation*}
\Vert v_n\Vert_{L^p(\Omega)} > n\Vert Dv_n\Vert_{L^p(\Omega)}.
\end{equation*}
We can normalize $v_n$, we obtain a sequence $(u_k)_{k\in\mathbb{N}}$ in $K$ such that
\begin{equation*}
\Vert Du_k\Vert_{L^p(\Omega)} < {\frac 1 n}.
\end{equation*}
This procedure is justified since $K$ is a cone.\\
\par As $(u_k)_{k\in\mathbb{N}}$ is bounded, there is a subsequence $(u_{k_j})_{j\in\mathbb{N}}$ converging to some $u$ in $L^p(\Omega)$. Since $K$ is closed, $u\in K$. Then its derivative is $0$ and by the assumption, $u=0$ but $\Vert u\Vert_{L^p(\Omega)}=1$. Thus a contradiction.

\section*{Exercise 4}

\subsection*{(i)}

Suppose if we have a weak solution, by integration by parts, we get

\begin{align*}
\int_U fv &=-\int_\Omega \Delta uvdx,\\
& =  \int_{\Omega}\nabla u\cdot\nabla vdx-\int_{\partial\Omega}{\frac {\partial u} {\partial \nu}}vdx ,\\
&= \int_{\Omega}\nabla u\cdot\nabla vdx-\int_{\partial\Omega}\beta uvdx .
\end{align*}

for any $v\in H^1(\Omega)$. We define a bilinear form $B(u,v):H^1(\Omega)\times H^1(\Omega)\to\mathbb{R}$ such that
\begin{equation*}
B(u,v) =  \int_{\Omega}\nabla u\cdot\nabla vdx-\int_{\partial\Omega}\beta uvdx.
\end{equation*}

By trace theorem, we have
\begin{align*}
|B(u,v)| &\leq \Vert Du\Vert_{L^2(\Omega)}\Vert Dv\Vert_{L^2(\Omega)}+\vert\beta\vert\Vert u\Vert_{L^2(\partial\Omega)}\Vert v\Vert_{L^2(\partial\Omega)},\\
&\leq \Vert Du\Vert_{L^2(\Omega)}\Vert Dv\Vert_{L^2(\Omega)}+\vert\beta\vert C^2\Vert u\Vert_{H^1(\Omega)}\Vert v\Vert_{H^1(\Omega)},\\
\end{align*}
for some $C>0$. By the definition, we have
\begin{equation*}
\Vert u\Vert_{H^1(\Omega)}^2 = \Vert Du\Vert_{L^2(\Omega)}^2+\Vert u\Vert_{L^2(\Omega)}^2.
\end{equation*}
Thus we conclude
\begin{equation*}
|B(u,v)|\leq (\vert\beta\vert C^2+1)\Vert u\Vert_{H^1(\Omega)}\Vert v\Vert_{H^1(\Omega)}
\end{equation*}
, and this satisfies the first condition for Lax-Milgram theorem.\\

\par In the case $\beta\leq 0$, we will prove that $B$ satisfies the second condition as well. In order to do so, we will derive a contradiction by assuming it does not satisfy the condition. If that is the case the it is equivalent to say that for any $n\in\mathbb{N}$, there is $u_n\in H^1(\Omega)$ such that

\begin{align*}
B(u_n,u_n)<{\frac 1 n}\Vert u_n\Vert_{H^1(\Omega)}^2.
\end{align*}
We can normalize $u_n$ and conclude that 
\begin{align*}
B(u_n,u_n)<{\frac 1 n}.
\end{align*}
Since $\partial \Omega$ is a unit circle around the center, it is Lipschitz. As the sequence is bounded, it contains a subsequence $(u_{n_k})_{k\in\mathbb{N}}$ that converges to $u$ in $L^2(\Omega)$ thus weak derivative $(Du_{n_k})_{k\in\mathbb{N}}$ converges weakly.\\
\par Now we derive a contradiction by 
\begin{equation*}
B(u_{n_k},u_{n_k}) = \Vert Du_{n_k}\Vert^2_{L^2(\Omega)}-\beta\int_{\partial\Omega} (Tu_{n_k})^2\to 0.
\end{equation*} 
This implies
\begin{equation*}
 \Vert Du\Vert^2_{L^2(\Omega)}=0
\end{equation*}
Therefore, $u$ is a constant. On the other hands, we have that
\begin{equation*}
u|_{\partial \Omega}=0.
\end{equation*}
This is a contradiction as $\Vert u\Vert_{H^1(\Omega)}=1$. We conclude that $B$ satisfies the conditions for Lax-Milgram theorem, therefore has a unique solution.

\subsection*{(ii)}

Let $(r,\theta)$ be the polar coordinate, then we have
\begin{equation*}
\Delta u = {\frac {\partial^2 u} {\partial r^2}}+{\frac 1 r}{\frac {\partial u} {\partial r}}+{\frac 1 {r^2}}{\frac {\partial^2 u} {\partial \theta^2}}.
\end{equation*}

Suppose $u$ is in the form $u(r,\theta) = R(r)A(\theta)$. Then we have
\begin{equation*}
\Delta u = (R''(r)+{\frac 1 r}R'(r))A(\theta)+R(r){\frac 1 {r^2}}A''(\theta) = 0.
\end{equation*}

Therefore, transforming the equation, we derive,
\begin{equation*}
{\frac {r^2R''(r)+rR'(r)} {R(r)}}=-{\frac {A''(\theta)} {A(\theta)}}.
\end{equation*}

Each side has a different variable. Thus this is equal to a constant $\lambda$. First obviously
\begin{equation*}
A(\theta) = a_1e^{\sqrt{\lambda}}+a_2e^{-\sqrt{\lambda}}.
\end{equation*}
for some constants, $a_1,a_2$. And for $R$,
\begin{equation*}
r^2R''(r)+rR'(r) =\lambda R(r)
\end{equation*}
By substituting $R(r) = r^\alpha$, we obtain
\begin{equation*}
\alpha(\alpha-1)+\alpha = \lambda.
\end{equation*}
Thus $\lambda = \alpha^2$.

First assumer $\lambda \geq 0$ then 
\begin{equation*}
R(r) = r^{\sqrt{\lambda}}, A(\theta) = a_1\cos(\sqrt{\lambda}\theta)+a_2\sin(\sqrt{\lambda}\theta).
\end{equation*}

For the boundary condition, we have $r=1$ and 
\begin{equation*}
{\frac {\partial u} {\partial r}} = {\frac {\partial u} {\partial x}}\cos\theta+{\frac {\partial u} {\partial y}}\sin\theta.
\end{equation*}
Substituting this to the condition we get,
\begin{align*}
R'(1)A(\theta) = \beta R(1)A(\theta).
\end{align*}

Where $R(1) = 1,R'(1) = \sqrt{\lambda}$. Thus for $\beta>0$,

\begin{equation*}
R(r)A(\theta) = r^{\beta}(a_1\cos(\beta\theta)+a_2\sin\beta\theta).
\end{equation*}

\end{document}