\documentclass{article}
\usepackage[utf8]{inputenc}
\usepackage{amsmath,amsfonts,amsthm,amssymb}
\usepackage{graphicx}
\usepackage{lmodern}
\usepackage[T1]{fontenc}
\usepackage{mathtools}
\usepackage{tikz-cd}
\usepackage[a4paper, total={6in, 8in}]{geometry}
\usepackage{xfrac}
\usepackage{enumitem}

\newcommand{\R}{\mathbb{R}}
\newcommand{\Z}{\mathbb{Z}}
\newcommand{\N}{\mathbb{N}}
\newcommand{\Q}{\mathbb{Q}}
\newcommand{\C}{\mathbb{C}}
\newcommand{\F}{\mathbb{F}}
\newcommand{\abs}[1]{\vert#1\vert}
\newcommand{\systeme}[2]{\begin{equation}
    \left\{
    \begin{aligned}
        &#1\\
        &#2\\
    \end{aligned}
    \right.
\end{equation}}
\newcommand{\limit}[2]{\lim_{#1\to#2}}
\newcommand{\power}[2]{\left(#1\right)^{#2}}
%L'accent aigu (é)
%L'accent grave (à, è, ù)
%L'accent circonflexe or "chapeau" (â, ê, î, ô, û)
%La cédille (ç)
%Le tréma (ë, ï, ü, ö)
\newtheorem{theorem}{Theorem}
\newtheorem{definition}{Definition}
\newtheorem{lemma}{Lemma}
\newtheorem{proposition}{Proposition}
\newtheorem{corollary}{Corollary}
\newtheorem{remark}{Remark}
\newtheorem{exercise}{Exercise}
\newtheorem{example}{Example}
\DeclareMathOperator{\Aut}{\textup{Aut}}
\DeclareMathOperator{\Frac}{\textup{Frac}}
\DeclareMathOperator{\ev}{\textup{ev}}
\DeclareMathOperator{\GL}{\textup{GL}}
\DeclareMathOperator{\PGL}{\textup{PGL}}
\DeclareMathOperator{\End}{\textup{End}}
\DeclareMathOperator{\SO}{\textup{SO}}
\DeclareMathOperator{\Sym}{\textup{Sym}}
\DeclareMathOperator{\tr}{\textup{tr}}
\DeclareMathOperator{\diag}{\textup{diag}}
\DeclareMathOperator{\diam}{\textup{diam}}
\DeclareMathOperator{\spt}{\textup{spt}}
\DeclareMathOperator{\loc}{\textup{loc}}
\DeclareMathOperator{\Lip}{\textup{Lip}}
\DeclareMathOperator{\Div}{\textup{div}}
\DeclareMathOperator*{\argmin}{arg\,min}
\DeclareMathOperator*{\argmax}{arg\,max}


\title{Sheet 1 Solutions}
\author{Toni Abi Aoun, So Murata}
\date{}

\begin{document}

\maketitle

\section*{Exercise 1}

Suppose the inequality does not hold. Then for each $n$, there is $v_n$ such that
\begin{equation*}
\Vert v_n\Vert_{L^p(\Omega)} > n\Vert Dv_n\Vert_{L^p(\Omega)}.
\end{equation*}
We can normalize $v_n$, we obtain a sequence $(u_k)_{k\in\mathbb{N}}$ in $K$ such that
\begin{equation*}
\Vert Du_k\Vert_{L^p(\Omega)} < {\frac 1 n}.
\end{equation*}
This procedure is justified since $K$ is a cone.\\
\par As $(u_k)_{k\in\mathbb{N}}$ is bounded, there is a subsequence $(u_{k_j})_{j\in\mathbb{N}}$ converging to some $u$ in $L^p(\Omega)$. Since $K$ is closed, $u\in K$. Then its derivative is $0$ and by the assumption, $u=0$ but $\Vert u\Vert_{L^p(\Omega)}=1$. Thus a contradiction.

\textbf{Problem 1.2}\\
Let $\Omega\subset\R^n$ be open with smooth boundary.\\
(i) Let $f\in L^2(\Omega)$ and show that the problem 
\[
\begin{cases}
    -\Delta w=f &\text{ in }\Omega,\\
    w=0 &\text{ on }\partial \Omega
\end{cases}
\]
has a unique weak solution $u\in H_0^1(\Omega)$ and that this solution satisfies for some constant $C>0$
\[
\Vert\nabla u\rVert_{L^2}\leqslant C\lVert f\rVert_{L^2}.
\]
\begin{proof}
    To derive the weak formulation of this problem, consider $\varphi\in C_c^\infty(\Omega)$
    \[
    \int_{\Omega}(-\Delta w)\varphi dx=\int_{\Omega}\nabla w\cdot\nabla\varphi dx=\int_\Omega f\varphi dx.
    \]
    So the weak formulation of this problem is to find $u\in H_0^1(\Omega)$ such that
    \[
    \int_{\Omega}\nabla u\cdot\nabla\varphi dx=\int_{\Omega}f\varphi dx\quad\forall \varphi\in C_c^\infty(\Omega).
    \]
    To this end consider the bilinear form $B:H_0^1(\Omega)\times H_0^1(\Omega)\to\R$ given by
    \[
    B(u,v)=\int_{\Omega}\nabla u\cdot\nabla vdx \quad\forall u,v\in H_0^1(\Omega),
    \]
    and the linear functional $f^*:H_0^1(\Omega)\to\R$ given by
    \[
    \langle f^*,v\rangle=\int_{\Omega}fvdx\quad\forall v\in H_0^1(\Omega).
    \]
    $B$ is clearly bilinear and  $\forall u,v\in H_0^1(\Omega)$, we have
    
    \begin{gather}
        |B(u,v)|=\left|\int_{\Omega}\nabla u\cdot \nabla vdx\right|\leqslant \lVert \nabla u\rVert_{L^2}\lVert \nabla v\rVert_{L^2}=\lVert u\rVert_{H_0^1}\lVert v\rVert_{H_0^1},\\
        |B(u,u)|=\int_{\Omega}|\nabla u|^2dx=\lVert u\rVert_{H_0^1}^2.
    \end{gather}
    This proves that $B$ is bounded and coercive.\\
    Similarly, for all $v\in H_0^1(\Omega)$, we have
    \[
    |\langle f^*,v\rangle|=\left|\int_{\Omega}fvdx\right|\leqslant \lVert f\rVert_{L^2}\lVert v\rVert_{L^2}\leqslant C_{Poi}\lVert f\rVert_{L^2}\lVert v\rVert_{H_0^1},
    \]
    where $C_{Poi}$ is the constant coming from Poincare's inequality, this proves that $f^*$ is a continuous linear functional on $H_0^1(\Omega)$.\\
    Since $H_0^1(\Omega)$ is a Hilbert space, we can apply Lax-Milgram's theorem to conclude that there exists a unique $u\in H_0^1(\Omega)$ such that 
    \[
    B(u,v)=\int_{\Omega}\nabla u\cdot\nabla vdx=\int_{\Omega}fvdx=\langle f^*,v\rangle \quad\forall v\in H_0^1(\Omega).
    \]
    This weak solution further satisfies
    \[
    \lVert \nabla u\rVert_{L^2}^2=\int_{\Omega}|\nabla u|^2dx=\int_{\Omega}fudx\leqslant \lVert f\rVert_{L^2}\lVert u\rVert_{L^2}\leqslant C_{Poi}\lVert f\rVert_{L^2}\lVert \nabla u\rVert_{L^2},
    \]
    which in turns yields
    \[
    \lVert \nabla u\rVert_{L^2}\leqslant C_{Poi}\lVert f\rVert_{L^2}.
    \]
\end{proof}
(ii) Let $f:\R\to\R$ be Lipschitz continuous. Show that
\[
\begin{cases}
    -\Delta u=f(u)&\text{ in }\Omega,\\
    u=0 &\text{ on }\partial \Omega
\end{cases}
\]
has a unique weak solution $u\in H_0^1(\Omega)$ provided that the Lipschitz constant of $f$ is sufficiently small.
\begin{proof}
    Since $f$ is Lipschtiz, for all $x\in\R$ we have
    \[
    |f(x)|\leqslant |f(0)|+\Lip(f)|x|.
    \]
    From this we deduce that for all $v\in L^2(\Omega)$, we have $f\circ v\in L^2(\Omega)$ since $\Omega$ is bounded (therefore of finite measure, so constant function are square integrable).\\
    Now using question (i), we can define the map $T:L^2(\Omega)\to L^2(\Omega)$, where for any $v\in L^2(\Omega)$, we have $T(v)$ is the unique weak solution in $H_0^1(\Omega)$ to the problem
    \[
    \begin{cases}
        -\Delta u=f(v)&\text{ in }\Omega,\\
        u=0&\text{ on }\partial\Omega.
    \end{cases}
    \]
    Given $v,w\in L^2(\Omega)$, we have that $T(v)-T(w)$ is the unique weak solution in $H_0^1(\Omega)$ to the problem
    \[
    \begin{cases}
        -\Delta u=f(v)-f(w)&\textup{ in }\Omega,\\
        u=0&\text{ on }\partial\Omega.
    \end{cases}
    \]
    Using the estimate of question $(i)$, we get
    \[
    \lVert\nabla T(v)-\nabla T(w)\rVert_{L^2}\leqslant C_{Poi}\lVert f(v)-f(w)\rVert_{L^2}\leqslant C_{Poi}\Lip(f)\lVert v-w\rVert_{L^2}.
    \]
    Using Poincare's inequality we thus get
    \[
    \lVert T(v)-T(w)\rVert_{L^2}\leqslant C_{Poi}\lVert \nabla T(v)-\nabla T(w)\rVert_{L^2}\leqslant C_{Poi}^2\Lip(f)\lVert v-w\rVert_{L^2}.
    \]
    $L^2(\Omega)$ is a Hilbert space thus a complete metric space. If $C_{Poi}^2\Lip(f)<1$, we can apply Banach's fixed point theorem, to deduce that there is a unique fixed point $u\in L^2(\Omega)$. Since $T(v)\in H_0^1(\Omega)$ for all $v\in L^2(\Omega)$, we get that $u=T(u)\in H_0^1(\Omega)$.
\end{proof}
\textbf{Problem 1.3}\\
Let $\Omega\subset\R^n$ be open and bounded with smooth boundary.\\
Recall the following version of the maximum/minimum principle: let $b,c\in C^0(\overline{\Omega}),\; c\geqslant 0$ and assume that $u\in C^2(\Omega)\cap C^0(\overline{\Omega})$ satsifies
\[
-\Delta u+b\cdot\nabla u+cu\geqslant 0 \text{ in }\Omega.
\]
Then 
\[
\min_{\overline{\Omega}}u\geqslant -\max_{\partial\Omega}u_-.
\]
(i) Show that the unique solution of 
\[
\begin{cases}
    \Delta u=u^3 &\text{ in }\Omega,\\
    u=0 &\text{ on }\partial\Omega
\end{cases}
\]
in $C^2(\Omega)\cap C^0(\overline{\Omega})$ is $u\equiv 0$.
\begin{proof}
    Set $b=0$ and $c=u^2\geqslant 0$, then $b,c\in C^0(\overline{\Omega})$ and we have
    \[
    -\Delta u+u^3=-\Delta u+b\cdot \nabla u+cu=0 \text{ in }\Omega.
    \]
    Using the maximum/minimum principle, we get that 
    \[
    \min_{\overline{\Omega}}u\geqslant-\max_{\partial\Omega}u_-=0.
    \]
    By applying the same reasoning to $-u$, we deduce that
    \[
    \max_{\overline{\Omega}}u=-\min_{\overline{\Omega}}(-u)\leqslant -\max_{\partial\Omega}(-u)_-=\min_{\partial\Omega}-u_+=0.
    \]
    Combining these two inequalities gives that $u\equiv 0$.
\end{proof}
(ii) Show that the problem
\[
\begin{cases}
    \Delta u=u^2&\text{ in }\Omega,\\
    u\geqslant 0 &\text{ in }\Omega,\\
    u=0&\text{ on }\partial\Omega
\end{cases}
\]
has a unique solution in $C^2(\Omega)\cap C^0(\overline{\Omega})$. Is the same true if the constraint $u\geqslant 0$ is dropped?
\begin{proof}
    Set $b=0$ and $c=u\geqslant 0$, then $b,c\in C^0(\overline{\Omega})$ and we have
    \[
    -\Delta(-u)+u(-u)=-\Delta(-u) +b\cdot\nabla(-u)+c(-u)=0\text{ in }\Omega.
    \]
    Using the maximum/minimum principle, we get that 
    \[
    \min_{\overline{\Omega}}(-u)\geqslant-\max_{\partial \Omega}(-u)_-,
    \]
    which is equivalent to
    \[
    \max_{\overline{\Omega}} u\leqslant \max_{\partial\Omega}u_+=0.
    \]
    Combined with the constraint $u\geqslant 0$ in $\Omega$, we get that the only solution in $C^2(\Omega)\cap C^0(\overline{\Omega})$ to the problem is $u\equiv 0$.\\
   
\end{proof}

\section*{Exercise 4}

\subsection*{(i)}

Suppose if we have a weak solution, by integration by parts, we get

\begin{align*}
\int_U fv &=-\int_\Omega \Delta uvdx,\\
& =  \int_{\Omega}\nabla u\cdot\nabla vdx-\int_{\partial\Omega}{\frac {\partial u} {\partial \nu}}vdx ,\\
&= \int_{\Omega}\nabla u\cdot\nabla vdx-\int_{\partial\Omega}\beta uvdx .
\end{align*}

for any $v\in H^1(\Omega)$. We define a bilinear form $B(u,v):H^1(\Omega)\times H^1(\Omega)\to\mathbb{R}$ such that
\begin{equation*}
B(u,v) =  \int_{\Omega}\nabla u\cdot\nabla vdx-\int_{\partial\Omega}\beta uvdx.
\end{equation*}

By trace theorem, we have
\begin{align*}
|B(u,v)| &\leq \Vert Du\Vert_{L^2(\Omega)}\Vert Dv\Vert_{L^2(\Omega)}+\vert\beta\vert\Vert u\Vert_{L^2(\partial\Omega)}\Vert v\Vert_{L^2(\partial\Omega)},\\
&\leq \Vert Du\Vert_{L^2(\Omega)}\Vert Dv\Vert_{L^2(\Omega)}+\vert\beta\vert C^2\Vert u\Vert_{H^1(\Omega)}\Vert v\Vert_{H^1(\Omega)},\\
\end{align*}
for some $C>0$. By the definition, we have
\begin{equation*}
\Vert u\Vert_{H^1(\Omega)}^2 = \Vert Du\Vert_{L^2(\Omega)}^2+\Vert u\Vert_{L^2(\Omega)}^2.
\end{equation*}
Thus we conclude
\begin{equation*}
|B(u,v)|\leq (\vert\beta\vert C^2+1)\Vert u\Vert_{H^1(\Omega)}\Vert v\Vert_{H^1(\Omega)}
\end{equation*}
, and this satisfies the first condition for Lax-Milgram theorem.\\

\par In the case $\beta\leq 0$, we will prove that $B$ satisfies the second condition as well. In order to do so, we will derive a contradiction by assuming it does not satisfy the condition. If that is the case the it is equivalent to say that for any $n\in\mathbb{N}$, there is $u_n\in H^1(\Omega)$ such that

\begin{align*}
B(u_n,u_n)<{\frac 1 n}\Vert u_n\Vert_{H^1(\Omega)}^2.
\end{align*}
We can normalize $u_n$ and conclude that 
\begin{align*}
B(u_n,u_n)<{\frac 1 n}.
\end{align*}
Since $\partial \Omega$ is a unit circle around the center, it is Lipschitz. As the sequence is bounded, it contains a subsequence $(u_{n_k})_{k\in\mathbb{N}}$ that converges to $u$ in $L^2(\Omega)$ thus weak derivative $(Du_{n_k})_{k\in\mathbb{N}}$ converges weakly.\\
\par Now we derive a contradiction by 
\begin{equation*}
B(u_{n_k},u_{n_k}) = \Vert Du_{n_k}\Vert^2_{L^2(\Omega)}-\beta\int_{\partial\Omega} (Tu_{n_k})^2\to 0.
\end{equation*} 
This implies
\begin{equation*}
 \Vert Du\Vert^2_{L^2(\Omega)}=0
\end{equation*}
Therefore, $u$ is a constant. On the other hands, we have that
\begin{equation*}
u|_{\partial \Omega}=0.
\end{equation*}
This is a contradiction as $\Vert u\Vert_{H^1(\Omega)}=1$. We conclude that $B$ satisfies the conditions for Lax-Milgram theorem, therefore has a unique solution.

\subsection*{(ii)}

Let $(r,\theta)$ be the polar coordinate, then we have
\begin{equation*}
\Delta u = {\frac {\partial^2 u} {\partial r^2}}+{\frac 1 r}{\frac {\partial u} {\partial r}}+{\frac 1 {r^2}}{\frac {\partial^2 u} {\partial \theta^2}}.
\end{equation*}

Suppose $u$ is in the form $u(r,\theta) = R(r)A(\theta)$. Then we have
\begin{equation*}
\Delta u = (R''(r)+{\frac 1 r}R'(r))A(\theta)+R(r){\frac 1 {r^2}}A''(\theta) = 0.
\end{equation*}

Therefore, transforming the equation, we derive,
\begin{equation*}
{\frac {r^2R''(r)+rR'(r)} {R(r)}}=-{\frac {A''(\theta)} {A(\theta)}}.
\end{equation*}

Each side has a different variable. Thus this is equal to a constant $\lambda$. First obviously
\begin{equation*}
A(\theta) = a_1e^{\sqrt{\lambda}}+a_2e^{-\sqrt{\lambda}}.
\end{equation*}
for some constants, $a_1,a_2$. And for $R$,
\begin{equation*}
r^2R''(r)+rR'(r) =\lambda R(r)
\end{equation*}
By substituting $R(r) = r^\alpha$, we obtain
\begin{equation*}
\alpha(\alpha-1)+\alpha = \lambda.
\end{equation*}
Thus $\lambda = \alpha^2$.

First assumer $\lambda \geq 0$ then 
\begin{equation*}
R(r) = r^{\sqrt{\lambda}}, A(\theta) = a_1\cos(\sqrt{\lambda}\theta)+a_2\sin(\sqrt{\lambda}\theta).
\end{equation*}

For the boundary condition, we have $r=1$ and 
\begin{equation*}
{\frac {\partial u} {\partial r}} = {\frac {\partial u} {\partial x}}\cos\theta+{\frac {\partial u} {\partial y}}\sin\theta.
\end{equation*}
Substituting this to the condition we get,
\begin{align*}
R'(1)A(\theta) = \beta R(1)A(\theta).
\end{align*}

Where $R(1) = 1,R'(1) = \sqrt{\lambda}$. Thus for $\beta>0$,

\begin{equation*}
R(r)A(\theta) = r^{\beta}(a_1\cos(\beta\theta)+a_2\sin\beta\theta).
\end{equation*}



\end{document}
