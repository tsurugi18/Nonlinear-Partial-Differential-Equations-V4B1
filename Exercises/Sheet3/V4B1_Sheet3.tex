\documentclass{article}
\usepackage[utf8]{inputenc}
\usepackage{amsmath,amsfonts,amsthm,amssymb}
\usepackage{graphicx}
\usepackage{lmodern}
\usepackage[T1]{fontenc}
\usepackage{mathtools}
\usepackage{tikz-cd}
\usepackage[a4paper, total={6in, 8in}]{geometry}
\usepackage{xfrac}
\usepackage{enumitem}

\newcommand{\R}{\mathbb{R}}
\newcommand{\Z}{\mathbb{Z}}
\newcommand{\N}{\mathbb{N}}
\newcommand{\Q}{\mathbb{Q}}
\newcommand{\C}{\mathbb{C}}
\newcommand{\F}{\mathbb{F}}
\newcommand{\abs}[1]{\vert#1\vert}
\newcommand{\systeme}[2]{\begin{equation}
    \left\{
    \begin{aligned}
        &#1\\
        &#2\\
    \end{aligned}
    \right.
\end{equation}}
\newcommand{\limit}[2]{\lim_{#1\to#2}}
\newcommand{\power}[2]{\left(#1\right)^{#2}}
%L'accent aigu (é)
%L'accent grave (à, è, ù)
%L'accent circonflexe or "chapeau" (â, ê, î, ô, û)
%La cédille (ç)
%Le tréma (ë, ï, ü, ö)
\newtheorem{theorem}{Theorem}
\newtheorem{definition}{Definition}
\newtheorem{lemma}{Lemma}
\newtheorem{proposition}{Proposition}
\newtheorem{corollary}{Corollary}
\newtheorem{remark}{Remark}
\newtheorem{exercise}{Exercise}
\newtheorem{example}{Example}
\DeclareMathOperator{\Aut}{\textup{Aut}}
\DeclareMathOperator{\Frac}{\textup{Frac}}
\DeclareMathOperator{\ev}{\textup{ev}}
\DeclareMathOperator{\GL}{\textup{GL}}
\DeclareMathOperator{\PGL}{\textup{PGL}}
\DeclareMathOperator{\End}{\textup{End}}
\DeclareMathOperator{\SO}{\textup{SO}}
\DeclareMathOperator{\Sym}{\textup{Sym}}
\DeclareMathOperator{\tr}{\textup{tr}}
\DeclareMathOperator{\diag}{\textup{diag}}
\DeclareMathOperator{\diam}{\textup{diam}}
\DeclareMathOperator{\spt}{\textup{spt}}
\DeclareMathOperator{\loc}{\textup{loc}}
\DeclareMathOperator{\Lip}{\textup{Lip}}
\DeclareMathOperator{\Div}{\textup{div}}
\DeclareMathOperator*{\argmin}{arg\,min}
\DeclareMathOperator*{\argmax}{arg\,max}

\title{Sheet 3 Solutions}
\author{Toni Abi Aoun, So Murata}
\date{}

\begin{document}
\maketitle

\textbf{Exercise 3.1}\\
Consider the one-dimensional wave equation for $(t,x)\in[0,\infty)\times\R$
\begin{equation}\label{wave}
    \partial_t^2u-\partial_x^2u=0,\quad u|_{t=0}=g,\quad \partial u_t|_{t=0}=f
\end{equation}
where $g,f:\R\to\R$ are given.\\
(i) Show that for smooth $g,f$ d'Alembert's formula
\[
u(t,x)=\frac{1}{2}(g(x-t)+g(x+t))+\frac{1}{2}\int_{x-t}^{x+t}f(y)dy
\]
yields a solution to \eqref{wave}.
\begin{proof}
    If $g\in C^2(\R)$ and $f\in C^1(\R)$, the above formula gives a solution. Indeed, in this case, we can differentiate the expression for $u$ using ordinary rules of differentiation and we get for $(t,x)\in (0,\infty)\times\R$
    \begin{gather*}
        \partial_xu(t,x)=\frac{1}{2}(g'(x-t)+g'(x+t))+\frac{1}{2}(f(x+t)-f(x-t))\\
        \partial_x^2u(t,x)=\frac{1}{2}(g''(x-t)+g''(x+t))+\frac{1}{2}(f'(x+t)-f'(x-t))\\
        \partial_tu(t,x)=\frac{1}{2}(g'(x+t)-g'(x-t))+\frac{1}{2}(f(x+t)+f(x-t))\\
        \partial_t^2u(x,t)=\frac{1}{2}(g''(x+t)+g''(x-t))+\frac{1}{2}(f'(x+t)-f'(x-t))
    \end{gather*}
    We see therefore that $u$ satisfies the one-dimensional wave equation in $(0,\infty)\times\R$ and its derivatives up to order 2 can be extended to $[0,\infty)\times\R$ since $g,f$ are uniformly continuous on compact sets. By directly evaluating the formula for $u$ and $\partial_tu$ at $(0,x)$, we see that $u$ satisfies the intial conditions.
\end{proof}
(ii) Let $f:[0,\infty)\times\R\to\R$ smooth and $v(t,x;s)$ be the solution to
\[
\partial_t^2v-\partial_x^2v=0,\quad v|_{t=s}=0,\quad\partial_tv|_{t=s}=f(s,\cdot).
\]
for all $s\geq 0$. Define
\[
u(t,x):=\int_0^tv(t,x;s)ds.
\]
Show that this provides a solution to the inhomogeneous wave equation
\begin{equation}\label{inwave}
    \partial_t^2u-\partial_x^2u=f,\quad u|_{t=0}=0,\quad \partial_tu|_{t=0}=0
\end{equation}
and compute a solution formula using (i).
\begin{proof}
    Suppose $f:[0,\infty)\times\R\to\R$ is $C^1$.\\
    For a fixed $s\geq 0$, the function $w_s:[0,\infty)\times\R\to\R$ defined by
    \[
    w_s(t,x)=v(s+t,x;s),\quad\forall (t,x)\in [0,\infty)\times\R
    \]
    solves the following one-dimensional wave equation on $[0,\infty)\times\R$
    \[
    \partial_t^2w_s-\partial_x^2w_s=0,\quad w_s|_{t=0}=0,\quad \partial_tw_s|_{t=0}=f(s,\cdot).
    \]
    From this using part (i), we deduce that 
    \[
    w_s(t,x)=\frac{1}{2}\int_{x-t}^{x+t}f(s,y)dy,\quad\forall (t,x)\in [0,\infty)\times\R.
    \]
    This gives the following formula for $v$ on $\{(t,x;s)|0\leq s\leq t, x\in\R\}$
    \[
    v(t,x;s)=w_s(t-s,x)=\frac{1}{2}\int_{x+s-t}^{x+t-s}f(s,y)dy.
    \]
    In particular $v$ is a $C^2$ function.\\
    Therefore $u$ is given by
    \[
    u(t,x)=\int_0^tv(t,x;s)ds=\frac{1}{2}\int_0^t\int_{x+s-t}^{x+t-s}f(s,y)dyds.
    \]
    To verify $u$ solves the inhomogeneous one-dimensional wave equation, we compute its derivatives
    \begin{gather*}
        \partial_tu(t,x)=v(t,x;t)+\int_{0}^t\partial_tv(t,x;s)ds=\int_0^t\partial_tv(t,x;s)ds\\
        \partial_t^2u(t,x)=\partial_tv(t,x;t)+\int_0^t\partial_t^2v(t,x;s)ds=f(t,x)+\int_0^t\partial_t^2v(t,x;s)ds\\
        \partial_xu(t,x)=\int_0^t\partial_xv(t,x;s)ds\\
        \partial_x^2u(t,x)=\int_0^t\partial_x^2v(t,x;s)ds=\int_0^t\partial_t^2v(t,x;s)ds. 
    \end{gather*}
    From this we see that for $(t,x)\in (0,\infty)\times\R$, we have
    \[
    \partial_t^2u(t,x)-\partial_x^2u(t,x)=f(t,x).
    \]
    For $x\in\R$, we have
    \[
    u(0,x)=0\text{ and }\partial_tu(0,x)=0
    \]
\end{proof}
(iii) Show that even if $f\in L^2_{\loc}((0,\infty)\times\R)$ the solution $u$ to \eqref{inwave} is in general not in $H^2_{\loc}((0,\infty)\times\R)$.
\begin{proof}
    
\end{proof}

\section*{3.2}

By the assumption, we have
\begin{equation}
\label{weak_sol}
\int_{\mathbb{R}^n} Du\cdot Dv dx = \int_{\mathbb{R}^n} fvdx - \int_{\mathbb{R}^n}c(u)vdx,
\end{equation}

for any $v\in H^1(\mathbb{R}^n)$. The above integration is defined since $u$ is compactly supported and we can choose large enough ball that contains the support of $u$ and integrate these expressions over it.\\
\par Let us now define 
\begin{equation*}
v = -D_k^{-h}(D_k^hu)
\end{equation*}
for sufficiently small $h$. Substitute this to Equation \ref{weak_sol}, we get
\begin{equation*}
-\int_{\mathbb{R}^n} Du\cdot D(D_k^{-h}(D_k^hu))dx = -\int_{\mathbb{R}^n} fD_k^{-h}(D_k^h u)dx + \int_{\mathbb{R}^n}c(u)D_k^{-h}(D_k^hu)dx.
\end{equation*}

By applying the integration by parts of difference quotients, we derive
\begin{equation*}
-\int_{\mathbb{R}^n} Du\cdot D(D_k^{-h}(D_k^hu))dx = \int_{\mathbb{R}^n} D_k^{-h}Du\cdot (D_k^hDu))dx=\Vert D_k^h(Du)\Vert^2_{L^2(\mathbb{R}^n)}.
\end{equation*}

By Cauchy inequality with $\varepsilon$ and the inequality between difference quotients and weak-derivatives, we get
\begin{align*}
\left|-\int_{\mathbb{R}^n} fD_k^{-h}(D_k^h u)dx\right| & \leq \int_{\mathbb{R}^n} |f||D_k^{-h}(D_k^h u)|dx\\
&\leq(\int_{\mathbb{R}^n} |f|^2dx)(\int_{\mathbb{R}^n}|D_k^{-h}(D_k^h u)|^2dx)\\
& \leq {\frac C \varepsilon}(\int_{\mathbb{R}^n} |f|^2dx)+\varepsilon(\int_{\mathbb{R}^n}|D_k^{-h}(D_k^h u)|^2dx)\\
& \leq {\frac C \varepsilon}(\int_{\mathbb{R}^n} |f|^2dx)+C_1\varepsilon(\int_{\mathbb{R}^n}|D_k^{h}(Du)|^2dx)
\end{align*}
Also we observe that by the smoothness of $c$,
\begin{equation*}
c(u)(x) = \int_0^{u(x)}c'(t)dt \Rightarrow |c(u)(x)|\leq |u(x)|\cdot\Vert c'\Vert_{L^\infty([0,u(x)])}.
\end{equation*}
With this we have the inequality and the same argument appeared previously we get 
\begin{equation*}
|\int_{\mathbb{R}^n}c(u)D_k^{-h}(D_k^hu)dx|\leq C_2\varepsilon\int_{\mathbb{R}^n}|D_k^h(Du)|^2dx+{\frac C \varepsilon}\Vert c'\Vert_{L^\infty([0,u(x)])}\Vert u\Vert^2_{L^2}.
\end{equation*}
Combining these inequalities and the rewriting the expression of the left hand side we get
\begin{equation*}
\Vert D_k^h(Du)\Vert^2_{L^2(\mathbb{R}^n)}\leq (C_1+C_2)\varepsilon\int_{\mathbb{R}^n}|D_k^h(Du)|^2dx + {\frac C \varepsilon}(\int_{\mathbb{R}^n} |f|^2dx) +{\frac C \varepsilon}\Vert c'\Vert_{L^\infty([0,u(x)])}\Vert u\Vert^2_{L^2}
\end{equation*}
Let $\varepsilon = {\frac 1 {2(C_1+C_2)}}$, then we get
\begin{equation*}
{\frac 1 2}\Vert D_k^h(Du)\Vert^2_{L^2(\mathbb{R}^n)}\leq{\frac C \varepsilon}(\int_{\mathbb{R}^n} |f|^2dx) +{\frac C \varepsilon}\Vert c'\Vert_{L^\infty([0,u(x)])}\Vert u\Vert^2_{L^2}
\end{equation*}
This holds for each $k=1,\cdots,n$, thus we conclude that for some constant $K$, the following inequality holds,
\begin{equation*}
\Vert D^hu\Vert_{L^2(\mathbb{R}^n)}\leq K((\int_{\mathbb{R}^n} |f|^2dx) +\Vert c'\Vert_{L^\infty([0,u(x)])}\Vert u\Vert^2_{L^2}).
\end{equation*}
Therefore $Du\in H^1(\mathbb{R}^n)$, therefore $u\in H^2(\mathbb{R}^n)$. 

\section*{Exercise 3.3}
Since $\phi$ is smooth, we have $\phi(u)\in H^1(U)$. Let us now define a bilinear form
\begin{equation*}
B[u,v] = \sum_{i,j}^n \int_{U} A_{i,j}\partial_{x_i}u\partial_{x_j}v.
\end{equation*}
Let $v\in\mathcal{C}^\infty_C(U)$ and $v\geq0$, then 
\begin{align*}
B[\phi(u),v] & = \sum_{i,j}^n \int_{U} A_{i,j}\partial_{x_i}\phi(u)\partial_{x_j}v\\
& = \sum_{i,j}^n \int_{U} A_{i,j}\phi'(u)\partial_{x_i}u\partial_{x_j}v\\
& = \sum_{i,j}^n \left(\int_{U} A_{i,j}\partial_{x_j}(\phi'(u)v)\partial_{x_i}udx-\int_{U} A_{i,j}\phi''(u)\partial_{x_i}u\partial_{x_j}u \cdot vdx\right).
\end{align*}
By uniform ellipticity and convexity of $\phi$ we get 
\begin{equation*}
\sum_{i,j}^n\int_{U} A_{i,j}\phi''(u)\partial_{x_i}u\partial_{x_j}u \cdot vdx \geq 0.
\end{equation*}
Also we have $u$ is the weak-solution of the original problem therefore
\begin{equation*}
\sum_{i,j}^n \left(\int_{U} A_{i,j}\partial_{x_j}(\phi'(u)v)\partial_{x_i}udx\right) = \phi'(u)v\sum_{i,j}^n \left(\int_{U} \partial_{x_j}(A_{i,j}\partial_{x_i}u)dx\right)=0.
\end{equation*}
Combining these we conclude that 
\begin{equation*}
B[\phi(u),v]\leq0.
\end{equation*}
By the density of test functions, we arrived the conclusion.

\textbf{Exercise 3.4}\\
Let $U\subset\R^n$ be an open bounded domain with smooth boundary. Consider the equation
\begin{equation}\label{bilaplace}
\begin{cases}
    \Delta^2u=f &\text{in }U,\\
    u=\frac{\partial u}{\partial\nu}=0&\text{on }\partial U.
\end{cases}
\end{equation}
We say that $u\in H_0^2$ is a weak solution to \eqref{bilaplace} provided
\[
\int_U\Delta u\Delta v=\int_Ufv
\]
for all $v\in H_0^2$. Given $f\in L^2(U)$, prove that there exists a unique weak solution to \eqref{bilaplace}.
\begin{proof}
    Consider the map $F:H_0^2(U)\to\R$ defined by
    \[
    F(v)=\int_Ufv,\quad\forall v\in H_0^2(U).
    \]
    This map is well defined (Cauchy-Schwarz), linear (linearity of integral) and bounded. To prove boundedness, we argue as follow
    \[
    |F(v)|=\left|\int_Ufv\right|\leq \lVert f\rVert_{L^2}\lVert v\rVert_{L^2}\leq \lVert f\rVert_{L^2}\lVert v\rVert_{H^2_0}.
    \]
    Now consider the map $B:H_0^2(U)\times H_0^2(U)\to\R$ defined by
    \[
    B(u,v)=\int_U\Delta u\Delta v,\quad\forall u,v\in H_0^2(U).
    \]
    This map is well defined (Cauchy-Schwarz), bilinear (linearity of weak derivatives and integral), bounded and coercive. To prove boundedness, we use Cauchy-Schwarz
    \[
    |B(u,v)|=\left|\int_U\Delta u\Delta v\right|\leq\lVert\Delta u\rVert_{L^2}\lVert\Delta v\rVert_{L^2}\leq \lVert u\rVert_{H^2_0}\lVert v\rVert_{H_0^2}.
    \]
    To prove coercivity, we need to prove that there exists $\alpha>0$ such that 
    \[
    B(u,u)\geq\alpha\lVert u\rVert_{H_0^2}^2,\quad\forall u\in H_0^2(U).
    \]
    This is equivalent to showing that there exists $C>0$ such that 
    \[
    \lVert u\rVert_{H_0^2}^2\leq C\int_{U}(\Delta u)^2,\quad\forall u\in H_0^2(U).
    \]
    $U$ has smooth boundary and $u\in H_0^1(U)$, then in particular we have $u\in H_0^2(U)$ so we can apply Poincare's inequality and get
    \[
    \int_U|u|^2\leq C\int_U|Du|^2.
    \]
    As $u\in H_0^2(U)$ we know that $\forall i=1,\dots,n$ we have $D_iu\in H_0^1(U)$, so we can apply Poincare's inequality and get
    \[
    \int_U|D_iu|^2\leq C\int_U|D(D_iu)|^2.
    \]
    Combining these two observations, we see that it suffices then to show that for any $i,j=1,\dots,n$ we have
    \[
    \int_U|u_{x_ix_j}^2|^2\leq C\int_U(\Delta u)^2.
    \]
    Using the density of $C_c^\infty(U)$ in $H_0^2(U)$ (by definition), it suffices to prove such a constant $C$ exists if $u\in C_c^\infty(U)$. In this case, we have
    \[
    \begin{aligned}
        \int_U(\Delta u)^2&=\int_U\sum_{i,j=1}^nu_{x_ix_i}u_{x_jx_j}\\
        &=-\int_U\sum_{i,j=1}^nu_{x_jx_ix_i}u_{x_j}\\
        &=\int_U\sum_{i,j=1}^n|u_{x_ix_j}|^2
    \end{aligned}
    \]
    where we used integration by parts twice. This proves the coercivity bound for $B$ and we can thus apply Lax-Milgram theorem to conclude the existence of a unique weak solution in $H_0^2$ to the problem.
\end{proof}

\end{document}