\documentclass{article}
\usepackage[utf8]{inputenc}
\usepackage{amsmath,amsfonts,amsthm,amssymb}
\usepackage{graphicx}
\usepackage{lmodern}
\usepackage[T1]{fontenc}
\usepackage{mathtools}
\usepackage{tikz-cd}
\usepackage[a4paper, total={6in, 8in}]{geometry}
\usepackage{xfrac}
\usepackage{enumitem}

\newcommand{\R}{\mathbb{R}}
\newcommand{\Z}{\mathbb{Z}}
\newcommand{\N}{\mathbb{N}}
\newcommand{\Q}{\mathbb{Q}}
\newcommand{\C}{\mathbb{C}}
\newcommand{\F}{\mathbb{F}}
\newcommand{\abs}[1]{\vert#1\vert}
\newcommand{\systeme}[2]{\begin{equation}
    \left\{
    \begin{aligned}
        &#1\\
        &#2\\
    \end{aligned}
    \right.
\end{equation}}
\newcommand{\limit}[2]{\lim_{#1\to#2}}
\newcommand{\power}[2]{\left(#1\right)^{#2}}
%L'accent aigu (é)
%L'accent grave (à, è, ù)
%L'accent circonflexe or "chapeau" (â, ê, î, ô, û)
%La cédille (ç)
%Le tréma (ë, ï, ü, ö)
\newtheorem{theorem}{Theorem}
\newtheorem{definition}{Definition}
\newtheorem{lemma}{Lemma}
\newtheorem{proposition}{Proposition}
\newtheorem{corollary}{Corollary}
\newtheorem{remark}{Remark}
\newtheorem{exercise}{Exercise}
\newtheorem{example}{Example}
\DeclareMathOperator{\Aut}{\textup{Aut}}
\DeclareMathOperator{\Frac}{\textup{Frac}}
\DeclareMathOperator{\ev}{\textup{ev}}
\DeclareMathOperator{\GL}{\textup{GL}}
\DeclareMathOperator{\PGL}{\textup{PGL}}
\DeclareMathOperator{\End}{\textup{End}}
\DeclareMathOperator{\SO}{\textup{SO}}
\DeclareMathOperator{\Sym}{\textup{Sym}}
\DeclareMathOperator{\tr}{\textup{tr}}
\DeclareMathOperator{\diag}{\textup{diag}}
\DeclareMathOperator{\diam}{\textup{diam}}
\DeclareMathOperator{\spt}{\textup{spt}}
\DeclareMathOperator{\loc}{\textup{loc}}
\DeclareMathOperator{\Lip}{\textup{Lip}}
\DeclareMathOperator{\Div}{\textup{div}}
\DeclareMathOperator*{\argmin}{arg\,min}
\DeclareMathOperator*{\argmax}{arg\,max}

\title{Sheet 3 Solutions}
\author{Toni Abi Aoun, So Murata}
\date{}

\begin{document}
\maketitle

\section*{3.2}

By the assumption, we have
\begin{equation}
\label{weak_sol}
\int_{\mathbb{R}^n} Du\cdot Dv dx = \int_{\mathbb{R}^n} fvdx - \int_{\mathbb{R}^n}c(u)vdx,
\end{equation}

for any $v\in H^1(\mathbb{R}^n)$. The above integration is defined since $u$ is compactly supported and we can choose large enough ball that contains the support of $u$ and integrate these expressions over it.\\
\par Let us now define 
\begin{equation*}
v = -D_k^{-h}(D_k^hu)
\end{equation*}
for sufficiently small $h$. Substitute this to Equation \ref{weak_sol}, we get
\begin{equation*}
-\int_{\mathbb{R}^n} Du\cdot D(D_k^{-h}(D_k^hu))dx = -\int_{\mathbb{R}^n} fD_k^{-h}(D_k^h u)dx + \int_{\mathbb{R}^n}c(u)D_k^{-h}(D_k^hu)dx.
\end{equation*}

By applying the integration by parts of difference quotients, we derive
\begin{equation*}
-\int_{\mathbb{R}^n} Du\cdot D(D_k^{-h}(D_k^hu))dx = \int_{\mathbb{R}^n} D_k^{-h}Du\cdot (D_k^hDu))dx=\Vert D_k^h(Du)\Vert^2_{L^2(\mathbb{R}^n)}.
\end{equation*}

By Cauchy inequality with $\varepsilon$ and the inequality between difference quotients and weak-derivatives, we get
\begin{align*}
\left|-\int_{\mathbb{R}^n} fD_k^{-h}(D_k^h u)dx\right| & \leq \int_{\mathbb{R}^n} |f||D_k^{-h}(D_k^h u)|dx\\
&\leq(\int_{\mathbb{R}^n} |f|^2dx)(\int_{\mathbb{R}^n}|D_k^{-h}(D_k^h u)|^2dx)\\
& \leq {\frac C \varepsilon}(\int_{\mathbb{R}^n} |f|^2dx)+\varepsilon(\int_{\mathbb{R}^n}|D_k^{-h}(D_k^h u)|^2dx)\\
& \leq {\frac C \varepsilon}(\int_{\mathbb{R}^n} |f|^2dx)+C_1\varepsilon(\int_{\mathbb{R}^n}|D_k^{h}(Du)|^2dx)
\end{align*}
Also we observe that by the smoothness of $c$,
\begin{equation*}
c(u)(x) = \int_0^{u(x)}c'(t)dt \Rightarrow |c(u)(x)|\leq |u(x)|\cdot\Vert c'\Vert_{L^\infty([0,u(x)])}.
\end{equation*}
With this we have the inequality and the same argument appeared previously we get 
\begin{equation*}
|\int_{\mathbb{R}^n}c(u)D_k^{-h}(D_k^hu)dx|\leq C_2\varepsilon\int_{\mathbb{R}^n}|D_k^h(Du)|^2dx+{\frac C \varepsilon}\Vert c'\Vert_{L^\infty([0,u(x)])}\Vert u\Vert^2_{L^2}.
\end{equation*}
Combining these inequalities and the rewriting the expression of the left hand side we get
\begin{equation*}
\Vert D_k^h(Du)\Vert^2_{L^2(\mathbb{R}^n)}\leq (C_1+C_2)\varepsilon\int_{\mathbb{R}^n}|D_k^h(Du)|^2dx + {\frac C \varepsilon}(\int_{\mathbb{R}^n} |f|^2dx) +{\frac C \varepsilon}\Vert c'\Vert_{L^\infty([0,u(x)])}\Vert u\Vert^2_{L^2}
\end{equation*}
Let $\varepsilon = {\frac 1 {2(C_1+C_2)}}$, then we get
\begin{equation*}
{\frac 1 2}\Vert D_k^h(Du)\Vert^2_{L^2(\mathbb{R}^n)}\leq{\frac C \varepsilon}(\int_{\mathbb{R}^n} |f|^2dx) +{\frac C \varepsilon}\Vert c'\Vert_{L^\infty([0,u(x)])}\Vert u\Vert^2_{L^2}
\end{equation*}
This holds for each $k=1,\cdots,n$, thus we conclude that for some constant $K$, the following inequality holds,
\begin{equation*}
\Vert D^hu\Vert_{L^2(\mathbb{R}^n)}\leq K((\int_{\mathbb{R}^n} |f|^2dx) +\Vert c'\Vert_{L^\infty([0,u(x)])}\Vert u\Vert^2_{L^2}).
\end{equation*}
Therefore $Du\in H^1(\mathbb{R}^n)$, therefore $u\in H^2(\mathbb{R}^n)$. 

\section*{Exercise 3.3}
Since $\phi$ is smooth, we have $\phi(u)\in H^1(U)$. Let us now define a bilinear form
\begin{equation*}
B[u,v] = \sum_{i,j}^n \int_{U} A_{i,j}\partial_{x_i}u\partial_{x_j}v.
\end{equation*}
Let $v\in\mathcal{C}^\infty_C(U)$ and $v\geq0$, then 
\begin{align*}
B[\phi(u),v] & = \sum_{i,j}^n \int_{U} A_{i,j}\partial_{x_i}\phi(u)\partial_{x_j}v\\
& = \sum_{i,j}^n \int_{U} A_{i,j}\phi'(u)\partial_{x_i}u\partial_{x_j}v\\
& = \sum_{i,j}^n \left(\int_{U} A_{i,j}\partial_{x_j}(\phi'(u)v)\partial_{x_i}udx-\int_{U} A_{i,j}\phi''(u)\partial_{x_i}u\partial_{x_j}u \cdot vdx\right).
\end{align*}
By uniform ellipticity and convexity of $\phi$ we get 
\begin{equation*}
\sum_{i,j}^n\int_{U} A_{i,j}\phi''(u)\partial_{x_i}u\partial_{x_j}u \cdot vdx \geq 0.
\end{equation*}
Also we have $u$ is the weak-solution of the original problem therefore
\begin{equation*}
\sum_{i,j}^n \left(\int_{U} A_{i,j}\partial_{x_j}(\phi'(u)v)\partial_{x_i}udx\right) = \phi'(u)v\sum_{i,j}^n \left(\int_{U} \partial_{x_j}(A_{i,j}\partial_{x_i}u)dx\right)=0.
\end{equation*}
Combining these we conclude that 
\begin{equation*}
B[\phi(u),v]\leq0.
\end{equation*}
By the density of test functions, we arrived the conclusion.
\end{document}