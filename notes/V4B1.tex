\documentclass{article}

\usepackage{amsmath}
\usepackage{amssymb}
\usepackage{amsthm}
\usepackage{enumerate}
\usepackage{bbm}
\usepackage{lipsum}

\newtheorem{theorem}{Theorem}[section] 
\newtheorem{proposition}{Proposition}[section] 
\newtheorem{definition}{Definition}[section] 
\newtheorem{lemma}{Lemma}[section] 
\newtheorem{notation}{Notation}[section] 
\newtheorem{remark}{Remark}[section] 
\newtheorem{corollary}{Corollary}[section] 
\newtheorem{terminology}{Terminology}[section] 
\newtheorem{example}{Example}[section] 
\numberwithin{equation}{section}

\DeclareMathOperator{\diam}{diam}
\DeclareMathOperator{\rank}{rank}
\DeclareMathOperator{\Hom}{Hom}
\DeclareMathOperator{\Dom}{Dom}
\DeclareMathOperator{\grad}{grad}
\DeclareMathOperator{\Span}{Span}
\DeclareMathOperator{\interior}{int}
\DeclareMathOperator{\ind}{ind}
\DeclareMathOperator{\supp}{supp}
\DeclareMathOperator{\ob}{ob}
\DeclareMathOperator{\Spec}{Spec}
\DeclareMathOperator{\PreSh}{PreSh}
\DeclareMathOperator{\Fun}{Fun}
\DeclareMathOperator{\esssup}{esssup}

\title{Nonlinear Partial Differential Equations 1}
\author{So Murata}
\date{2024/2025 Winter Semester - Uni Bonn}

\begin{document}
\maketitle

\section{Preliminaries}

\subsection{Measure theory}

\subsection{Sobolev spaces}

\begin{definition}
A $n$-dimensional multi-index is a $n$-tuple in $\mathbb{N}_0^n$. And for such $n$-tuple $\alpha$ we define $|\alpha| = \sum_{i=1}^n \alpha_i$. 
\end{definition}

\begin{definition}
Let $U$ be an open subset of $n$-dimensional real Euclidean space. A measurable function $f:U\to\mathbb{C}$ is said to be locally integrable over $U$ if for any compact subset $K$ of $U$, the integral $\int_K|f|dx$ is finite. The set of all such functions is denoted by $L^1_{\mathbf{loc}}(U)$.
\end{definition}

\begin{definition}
Let $X$ be a topological space. $f:X\to\mathbb{R}$ is said to be compactly supported if there is a compact subset $K$ of $X$ such that $f(X-K) = \{0\}$. 
\end{definition}

\begin{definition}
Let $U$ be an open subset of $\mathbb{R}^n$. A test function $\phi:U\to\mathbb{R}$ is a function such that it is infinitely continuously differentiable and compactly supported. The set of all such functions over $U$ is denoted by $\mathcal{C}^\infty_C(U)$.
\end{definition}

\begin{definition}
Let $u:U\to\mathbb{R}$ be a $|\alpha|$ times continuously differentiable function over an open subset $U$ of $\mathbb{R}^n$. Then the partial-derivative respect to $\alpha$ is 
\begin{equation*}
D^\alpha u= {\frac {\partial^{|\alpha|}u} {\partial x_1^{\alpha_1}\cdots\partial x_n^{\alpha_n}}}.
\end{equation*}
\end{definition}

\begin{definition}
Let $u,v\in L^1_{\mathbf{loc}}(U)$ where $U$ is an open subset of $\mathbb{R}^n$ and $\alpha$ be a multi-index. We say that $\alpha$-th weak derivative of $u$ is $v$(denoted as $D^\alpha u=v$), if the following equality holds.
\begin{equation*}
\int_U uD^\alpha\phi dx = (-1)^{|\alpha|}\int_v\phi dx
\end{equation*}
for any test function $\phi$. 
\end{definition}

\begin{remark}
In the definition above, if $|\alpha|=1$, this means the integration by parts.
\end{remark}

\begin{lemma}
Weak derivatives are unique almost everywhere. In other words, if $v_1,v_2$ are weak derivatives for $u$, then for any $\phi\in\mathcal{C}^\infty_C(U)$ we have
\begin{equation*}
\int_U (v_1-v_2)\phi dx = 0.
\end{equation*}
\end{lemma}

\begin{proof}
The above equality is clear as the left-hand side equals to 
\begin{equation*}\int_U uD^\alpha \phi dx-\int_U uD^\alpha \phi dx.
\end{equation*}
\end{proof}

\begin{definition}
Let $U$ be an open subset of $\mathbb{R}^n$, $k\in\mathbb{N}$, $p\in[1,\infty]$. We define the Sobolev space with $k$ and $p$ over $U$ such that 
\begin{equation*}
W^{k,p}(U) = \{u\in L^1_{\mathbf{loc}}(U)\:|\: \forall \alpha\in\mathbb{N}_0^n, |\alpha|\leq k \Rightarrow D^\alpha u\in L^p(U)\}.
\end{equation*}
\end{definition}

\begin{notation}
For the case $p=2$ we denote $W^{k,2}(U) = H^k(U)$. 
\end{notation}

\begin{remark}
When $k=0$ we have $W^{0,p}(U) = L^p(U)$.
\end{remark}

\begin{definition}
The essential supremum of a functional $f:U\to\mathbb{R}$ is 
\begin{equation*}
\esssup_U(f)\inf\{c\in\mathbb{R}\:|\: \forall x\in U, f(x)\leq c\}.
\end{equation*}
\end{definition}

\begin{definition}
We define a norm on a Sobolev space $W^{k,p}(U)$ such that for $u\in W^{k,p}(U)$,
\begin{equation*}
\Vert u \Vert_{W^{k,p}(U)} = 
\begin{cases}
\bigg{(}\sum_{|\alpha|\leq k} \int_U|D^\alpha u|^pdx\bigg{)}^{{\frac 1 p}}\quad(1\leq p<\infty),\\
\\
\sum_{|\alpha|\leq k}\esssup_U(|D^\alpha u|)\quad(p=\infty).
\end{cases}
\end{equation*}
\end{definition}

\begin{definition}
A sequence in the Sobolev space $W^{k,p}(U)$ converges if it converges in its Sobolev norm.
\end{definition}

\begin{definition}
For a sequence $(u_m)\subset W^{k,p}(U)$, $u_m\to u$ in $W_{\mathbf{loc}}^{k,p}(U)$ if for any compact subset $K$ of $U$, we have $u_m\to u$ in $W_{\mathbf{loc}}^{k,p}(K)$.
\end{definition}

\begin{definition}
The Sobolev space $W^{k,p}_0(U)$ with boundary value zero at $\partial U$  is the closure of $C^{\infty}_C(U)$ in the topology induced by the Sobolev norm.
\end{definition}

\subsection{Example}

Let $U=B(0,1)$ be a unit ball in $\mathbb{R}^n$ and $u:U\to\mathbb{R}$ be such that $u(x) = \Vert x \Vert ^{-\alpha}$ for any $x$ except $0$. Then we have the following statement,
\begin{equation*}
u\in W^{1,p}(U)\Leftrightarrow \alpha<{\frac {n-p} {p}}.
\end{equation*}

Let us consider $\{r_k\}_{k\in\mathbb{N}}\subset B(0,1)=U$ such that it is dense in $U$. The function 
\begin{equation*}
u(x) = \sum_{k=1}^\infty {\frac 1 {2^l}}|x-r_k|^{-\alpha},
\end{equation*}

is in $W^{1,p}(U)$ if and only if $\alpha={\frac {n-p} {p}}$. This function is not bounded in any ball contained in $U$. (Use the fact that $p<n$) but $u\not\in L^\infty(U)$. 

\begin{theorem}
Let $U$ be an open and bounded set in $\mathbb{R}^n$. Suppose that $u\in W^{k,p}(U)$, then there exists a sequence $(u_m)_{m\in\mathbb{N}}\subset C^\infty(U)\cap W^{k,p}(U)$ such that $u_m\to u$ in $W^{k,p}(U)$. 
\end{theorem}

\subsection{Sobolev Inequalities}

\begin{definition}
Let $p\in[1,\infty)$. The Soolev conjugate of $p$ is 
\begin{equation*}
p^* = {\frac {np} {n-p}}.
\end{equation*} 
\end{definition}

\begin{remark}
From the definition we see 
\begin{equation*}
{\frac 1 {p^*}} = {\frac 1 p}-{\frac 1 n}.
\end{equation*}
Thus $p^*>p$.
\end{remark}

\begin{theorem}
\label{poincare_ineq}
Let $U\subset \mathbb{R}^n$ be an open bounded set and $u\in W^{1,p}(U)$. Then there exists a constant $C$ only depending on $n$ and $U$ such that
\begin{equation*}
\Vert u - (u)_U\Vert_{L^2(U)}\leq C(n,U)\Vert Du\Vert|_{L^2(U)},
\end{equation*}
where 
\begin{equation*}
(u)_U={\frac 1 {\mu(U)}}\int_U udx.
\end{equation*}
\end{theorem}

\begin{corollary}
In Theorem \ref{poincare_ineq}, if we have $u\in H_0^1(U)$, then there exists a constant $C$ only depending on $n$ and $U$ such that
\begin{equation*}
\Vert u \Vert_{L^2(U)}\leq C(n,U)\Vert Du\Vert|_{L^2(U)},
\end{equation*}
\end{corollary}

\begin{remark}
We may now replace the Sobolev norm in $H^1_0(U)$ with $\Vert Du\Vert_{L^2(U)}$. Recall that the norm defined on $H^1_0(U)$ is 
\begin{equation*}
\Vert u\Vert_{H^1_0(U)} = \Vert u\Vert_{L^2(U)}+\Vert Du\Vert_{L^2(U)}.
\end{equation*}
By the corollary, we obtain the inequality,
\begin{equation*}
\Vert Du\Vert|_{L^2(U)}\leq\Vert u\Vert_{H^1_0(U)} \leq C(n,U)\Vert Du\Vert|_{L^2(U)}.
\end{equation*}
Therefore, $\Vert Du\Vert|_{L^2(U)}$ induces same topology as the Sobolev norm.
\end{remark}

\section{Second Order Elliptic Equations}

\subsection{Definitions}

\begin{notation}
For a function $u:U\to\mathbb{R}$ where $U$ is an open subset of $\mathbb{R}^n$, we use the following notation.
\begin{equation*}
{\frac {\partial u} {\partial x_i}} = u_{x_i}.
\end{equation*}
\end{notation}

\begin{definition}
Give an open bounded subset $U$ of $\mathbb{R}^n$, $f:U\to\mathbb{R}$. The second order elliptic equation is the problem to find functions $u:\overline{U}\to\mathbb{R}$ which satisfies the following equations,
\begin{equation*}
\begin{cases}
Lu(x)=f(x)\quad (x\in U),\\
u(x)\equiv 0 \quad(x\in\partial U),
\end{cases}
\end{equation*}

where $L$ is called a second-order partial differential operator such that for any $u:U\to\mathbb{R}$, 
\begin{equation}
\label{eq:2.1}
Lu = -\sum_{i,j=1}^n (a^{i,j}(x)u_{x_i})_{x_j}+\sum_{i=1}^nb^i(x)u_{x_i}+c(x)u,
\end{equation}
where for each $i,j=1,\cdots,n$, $a^{i,j},b^i,c:U\to\mathbb{R}$. 
\end{definition}

\begin{remark}
If $a^{i,j}$ are differentiable on $U$ for each $i,j=1,\cdots,n$, we can rewrite Equation \ref{eq:2.1} to be such that
\begin{equation}
\label{eq:2.2}
Lu = -\sum_{i,j=1}^n a^{i,j}(x)u_{x_ix_j}+\sum_{i=1}^n\overline{b}^i(x)u_{x_i}+c(x)u,
\end{equation}
where for each $i=1,\cdots,n$,
\begin{equation*}
\overline{b}^i(x) = b_i(x)-\sum_{j=1}^n (a^{i,j}(x))_{x_j}(x).
\end{equation*}
This is due to the Leibniz rule.
\end{remark}

\begin{remark}
In the case $u\in\mathcal{C}^2(U)$, we may assume that $a_{i,j}=a{j,i}$ for each $i,j=1,\cdots,n$ from now on. This is justified by the following procedure.
Given $(a^{i,j})_{i,j=1\cdots,n}$ we define $(\tilde{a}^{i,j})_{i,j=1\cdots,n}$ in the following way,
\begin{equation*}
\tilde{a}^{i,j}(x) = {\frac 1 2}(a^{i,j}(x)+a^{j,i}(x)).
\end{equation*}
Because the first part of the equation \ref{eq:2.1} can be rewritten as
\begin{equation*}
\sum_{i,j=1}^n {\frac 1 2}(a^{i,j}(x)+a^{j,i}(x))u_{x_ix_j}+\sum_{i,j=1}^n{\frac 1 2}(a^{i,j}(x)-a^{j,i}(x))u_{x_ix_j}.
\end{equation*}
Using Young's theorem we derive that
\begin{equation*}
\sum_{i,j=1}^n{\frac 1 2}(a^{i,j}(x)-a^{j,i}(x))u_{x_ix_j} = 0.
\end{equation*}
\end{remark}

\begin{definition}
A second-order partial differential operator is said to be uniformly elliptic if there is $\theta>0$ such that for any $\xi\in\mathbb{R}^n$
\begin{equation*}
\sum_{i,j=1}^n a^{i,j}(x)\xi_i\xi_j \geq \theta\Vert \xi\Vert^2 
\end{equation*}
holds for almost everywhere on $U$.
\end{definition}

\begin{remark}
The above definition can be stated in a different manner. Given a quadratic form $A(x) = (a^{i,j}(x))_{i,j=1,\cdots,n}$. The problem is uniformly elliptic if and only if 
\begin{equation*}
A(x)\geq \theta I
\end{equation*}
holds almost everywhere for a fixed constant $\theta>0$.
\end{remark}

\begin{example}
If we take $a^{i,j}=\delta_{i,j}$, and $b^i,c\equiv 0$, the problem is 
\begin{equation*}
Lu = -\Delta u.
\end{equation*}
\end{example}

\subsection{Weak Solutions}

In this subsection, we assume that $a^{i,j},b^i,c\in L^\infty(U)$ for each $i,j=1,\cdots,n$ and $f\in L^2(U)$. Suppose we have a second-order elliptic equation. Then by multiplying $v\in\mathcal{C}^\infty_0(U)$, we get
\begin{equation*}
-\int_U\sum_{i,j=1}^n(a^{i,j}(x)u_{x_i})_{x_j}vdx = \int_U\sum_{i,j=1}^na^{i,j}(x)u_{x_i}v_{x_j}dx
\end{equation*}
which is well-defined fi $\Vert Dv\Vert\in L^1(U)$. 

\begin{definition}
Given a second-order elliptic equation, we define a bilinear form $B:H_0^1(U)\times H_0^1(U)\to\mathbb{R}$ such that

\begin{equation*}
B(u,v) = \int_U\sum_{i,j=1}^na^{i,j}(x)u_{x_i}v_{x_j}+\sum_{i=1}^nb^i(x)u{x_i}v+c(x)uvdx
\end{equation*}
\end{definition}

\begin{remark}
Such $B(u,v)$ is a well-defined continuous bilinear form.
\end{remark}

\begin{definition}
Given a second-order elliptic equation
\begin{equation*}
\begin{cases}
Lu(x) = f(x) \quad (x\in U),\\
u(x) = 0 \quad (x\in \partial U).
\end{cases}
\end{equation*}
A function $u\in H_0^1(U)$ is called a weak solution to the problem if for any $v\in H_0^1(U)$, we have
\begin{equation*}
B(u,v) = \langle f,v\rangle_{L^2(U)}.
\end{equation*}
\end{definition}

\begin{remark}
Suppose we have a classical solution $u$, (in other words $u\in\mathcal{C}^2(U)$ and $a^{i,j}\in C^1(U)$). Then such $u$ is also a weak solution. 
\end{remark}

\begin{remark}
Suppose for $u\in H_0^1(U)$, we have that for any $v\in\mathcal{C}_0^\infty(U)$
\begin{equation*}
B(u,v) = \langle f,v\rangle_{L^2(U)}.
\end{equation*}
Then such $u$ is a weak-solution, as $\mathcal{C}_0^\infty(U)$ is dense in $H_0^1(U)$.
\end{remark}

We could also replace the condition on $f$ which is that $f\in L^2(U)$ to $f\in H^{-1}(U)$. 

\begin{definition}
Given a second-order elliptic equation, we say that $u\in H_0^1(U)$ is a weak solution of the problem if
\begin{equation*}
B[u,v]=\langle f,v\rangle,
\end{equation*}
where
\begin{equation*}
\langle f,v\rangle = \int_U f^0v+\sum_{i=1}^n f^iv_{x_i}dx\quad(f^0,f^1,\cdots,f^n\in L^2(U)),
\end{equation*}
is the duality pairing of $H^{-1}(U)$ and $H_0^1(U)$.
\end{definition}

\begin{proposition}
\par Suppose we have a problem
\begin{equation*}
\begin{cases}
Lu(x) = f(x)\quad(x\in U),\\
u(x) = g(x)\quad(x\in\partial U).
\end{cases}
\end{equation*}
where $\partial U$ is smoothly parametrizable. Furthermore, suppose there is $w\in H^1(U)$ such that 
\begin{equation*}
w(x) = g(x) \quad (x\in\partial U).
\end{equation*}
Then for the modified problem,
\begin{equation*}
\begin{cases}
L(u(x)) = f(x)-Lw(x)\quad(x\in U),\\
u(x) = 0\quad(x\in\partial U).
\end{cases}
\end{equation*}
we can get solutions of the original problem given a solution of the second one and adding $w$ to it.
\end{proposition}

\begin{proof}
In order to show that such modified problem is indeed well-defined, we have to prove that
\begin{equation*}
f-Lw\in H^{-1}(U).
\end{equation*}
By definition, $w\in H^1(U)$ thus $w_{x_i}\in L^2(U)$ for any $i=1,\cdots,n$. For any $i,j=1,\cdots,n, a^{i,j}\in L^\infty(U)$. We now deduce that $a^{i,j}w_{x_i}\in L^2(U)$. Later.
\end{proof}

\subsection{Existence of Weak Solutions}

\begin{theorem}[Lax-Milgram]
\label{LM}
Let $(H,\langle\cdot,\cdot\rangle_H)$ be a real Hilbert space and $H^*$ be the dual of it. Assume that for a bilinear form $B:H\times H\to\mathbb{R}$, there exists $\alpha,\beta>0$ such that
\begin{enumerate}[i).]
\item $\vert B(u,v)\vert\leq \alpha\Vert u\Vert\Vert v\Vert$,
\item for any $v\in H$, $\beta\Vert u\Vert^2\leq B(u,v)$.
\end{enumerate}
Then for each $f\in H^*$, there is a unique $u\in H$ such that
\begin{equation*}
B(u,v) = \langle f,v\rangle_H.
\end{equation*}
\end{theorem}

\begin{proof}
Given $u\in H$, the mapping $v\mapsto B(u,v)$ is a bounded linear operator by the first condition on $B$. By Riesz representation theorem, there is unique $w\in H$ such that 
\begin{equation*}
B(u,v) = \langle w,v\rangle_H.
\end{equation*}
Let us define $A:H\to H$ to be such that $A(u) = w$ where $w$ is acquired through the above construction. We will show that $A$ is a bounded linear operator.\\
\par We first show that it is linear. Given $\lambda_1,\lambda_2\in \mathbb{R}$ and $u_1,u_2\in H$,
\begin{align*}
\langle A(\lambda_1u_1+\lambda_2u_2,)v\rangle_H & = B(\lambda_1u_1+\lambda_2u_2,v)\\
& = \lambda_1B(u_1,v)+\lambda_2B(u_2,v)\\
& = \lambda_1\langle Au_1,v\rangle_H+\lambda_2\langle Au_2,v\rangle_H\\
& = \langle A\lambda_1u_1+A\lambda_2u_2,v\rangle_H
\end{align*}
holds for any $v\in H$. Thus by uniqueness from Riesz representation theorem, we have proved that $A$ is linear.\\
\par We then prove that $A$ is bounded.
\begin{align*}
\Vert Au\Vert^2& = \langle Au,Au\rangle_H,\\
& = B(u,Au),\\
& \leq \alpha\Vert u\Vert\Vert Au\Vert,\\
&\Rightarrow \Vert Au\Vert\leq \alpha\Vert u\Vert,
\end{align*}
by the first condition on $B$.\\
\par Such $A$ has following two properties.
\begin{enumerate}[i).]
\item $A$ is injective,
\item $\mathcal{R}(A)$, the range of $A$ is closed.
\end{enumerate}

Suppose $Au=0$ then by the second condition on $B$ we derive
\begin{align*}
\beta\Vert u\Vert^2 &\leq B(u,u),\\
& = \langle Au,u\rangle_H,\\
& \leq \Vert Au\Vert\Vert u\Vert\\
&\Rightarrow \beta\Vert u\Vert\leq\Vert Au\Vert = 0.
\end{align*}

Let $(Au_n)_{n\in\mathbb{N}}\subset\mathcal{R}(A)$ be a convergent sequence with its limit $w^*$. Using the second condition on $B$ and the previous argument on the norm of $Au$ once again, we derive
\begin{align*}
\beta\Vert u_n-u_m\Vert&\leq\Vert A(u_n-u_m)\Vert,\\
& =\Vert Au_n-Au_m\Vert.
\end{align*}
Sincer $(Au_n)_{n\in\mathbb{N}}$ is a Cauchy sequence we derived that $(u_n)_{n\in\mathbb{N}}$ is also a Cauchy sequence in a complete space, thus convergent. We define the limit to be $u^*$. By continuity of $A$, we have $Au^*= w^*$.\\
\par We now prove that $\mathcal{R}(A)=H$. Suppose not, $\mathcal{R}(A)\not=H$, then we know that $\mathcal{R}(A)$ is closed therefore
\begin{equation*}
(M^\perp)^\perp = M\not=H.
\end{equation*}
This shows that $M^\perp\not=\{0\}$. Take $u^\perp\in\mathcal{R}(A)^\perp$ and $u=Au^\perp$ which is in $\mathcal{R}(A)$. Then we have
\begin{equation*}
\beta\Vert u^\perp\Vert^2 \leq B(u^\perp,u^\perp)=\langle Au^\perp,u^\perp\rangle_H=0.
\end{equation*} 
Therefore, we derived $\Vert u^\perp\Vert = 0$ which is a contradiction. We conclude $\mathcal{R}(A)=H$.\\
\par Using Riesz representation theorem, there exists $w\in H$ such that for any $v\in H$,
\begin{equation*}
\langle f, v\rangle_H=\langle w,v\rangle_H.
\end{equation*}
By the subjectivity of $A$, there is $u\in H$ such that $Au=w$. Therefore we derive the formula,
\begin{equation*}
B(u,v) = \langle Au,v\rangle_H= \langle w,v\rangle_H = \langle f, v\rangle_H.
\end{equation*}
Now we will prove the uniqueness of such $u$. Suppose $u,\overline{u}\in H$ are such that
\begin{equation*}
 \langle u,v\rangle_H=\langle f, u\rangle_H=\langle \overline{u},v\rangle_H.
\end{equation*}
Then by the linearity of the scalar product, we obtain that for any $v\in H$,
\begin{equation*}
B(u-\overline{u},v)=0.
\end{equation*}
In particular, when $v=u-\overline{u}$, we obtain,
\begin{equation*}
\beta\Vert u-\overline{u}\Vert^2=0.
\end{equation*}
Thus the uniqueness is proven.
\end{proof}

\begin{remark}
$B$ does not have to be symmetric. In the case when $B$ is symmetric, the theorem is trivial.\\
\par This comes from by defining $(u,v)=B(u,v)$ a new scalar product which is possible since $B$ is symmetric and by the second condition we have
\begin{equation*}
(u,u)\geq\beta\Vert u\Vert^2\Rightarrow ((u,u)=0\Leftrightarrow u=0).
\end{equation*}
Using the Riesz representation theorem for $(\cdot,\cdot)$, we obtain that, for any $v\in H$, 
\begin{equation*}
(w,v) = \langle f, v\rangle_H.
\end{equation*}
\end{remark}

\begin{theorem}
Give an open bounded subset $U$ of $\mathbb{R}^n$, $f:U\to\mathbb{R}$. We consider the second order elliptic equation.
\begin{equation*}
\begin{cases}
Lu(x)=f(x)\quad (x\in U),\\
u(x)\equiv 0 \quad(x\in\partial U),
\end{cases}
\end{equation*}

where $L$ is a second-order partial differential operator such that for any $u:U\to\mathbb{R}$, 
\begin{equation}
\label{eq:2.1}
Lu = -\sum_{i,j=1}^n (a^{i,j}(x)u_{x_i})_{x_j}+\sum_{i=1}^nb^i(x)u_{x_i}+c(x)u,
\end{equation}
where for each $i,j=1,\cdots,n$, $a^{i,j},b^i,c:U\to\mathbb{R}$ belong to $L^\infty(U)$. 
Furthermore, we pose the uniformly elliptic condition to $L$.\\
\par Let us define $B(u,v):H_0^1(U)\times H_0^1(U)\to\mathbb{R}$ to be such that
\begin{equation*}
B(u,v) = \int_{U} \sum_{i,j=1}^n a^{i,j}u_{x_i}v_{x_j}+\sum_{i=1}^n b^iu_{x_i}v+cuv\:dx,
\end{equation*}
where $a^{i,j},b^i,c\in L^\infty(U)$.
Then there exists $\alpha,\beta>0$ and $\gamma\geq 0$ such that,
\begin{enumerate}[1).]
\item $B(u,v)\leq\alpha\Vert u \Vert_{H_0^1(U)}\Vert v\Vert_{H_0^1(U)}$.
\item $\beta\Vert u \Vert_{H_0^1(U)}^2 \leq B(u,u)+\gamma\Vert u \Vert_{L^2(U)}^2$.
\end{enumerate}
\label{LM_modified}
\end{theorem}

\begin{proof}
\begin{align*}
\vert B(u,v)\vert \leq & \sum_{i,j=1}^n \Vert a^{i,j}\Vert_{L^2(U)}\int_U \vert Du\vert\vert Dv\vert dx\\
& +\sum_{i=1}^n\Vert b^i\Vert_{L^\infty(U)}\int_U\vert Du\vert\vert v\vert dx\\
& + \Vert c\Vert_{L^\infty(U)}\int_U\vert u\vert\vert v\vert dx.
\end{align*}
We then use the Cauchy-Schwarz inequality to deduce that
\begin{align*}
\int_U \vert Du\vert\vert Dv\vert dx&\leq \Vert Du\Vert_{L^2(U)}\Vert Dv\Vert_{L^2(U)},\\\
\int_U\vert Du\vert\vert v\vert dx &\leq \Vert Du\Vert_{L^2(U)}\Vert v\Vert_{L^2(U)},\\
\int_U\vert u\vert\vert v\vert dx & \leq \Vert u\Vert_{L^2(U)}\Vert v\Vert_{L^2(U)}.
\end{align*}
Using Poincaré inequality, we derive that
\begin{equation*}
B(u,v)\leq \alpha \Vert u\Vert_{H_0^1(U)}\Vert v\Vert_{H_0^1(U)}.
\end{equation*}
\par We now prove the second claim. First we note that, by the uniformly elliptic condition, there is $\theta>0$ such that 
\begin{align*}
\theta\int_U\vert Du\vert^2&\leq\int_U\sum_{i,j=1}^n a^{i,j}u_{x_i}u_{x_j}dx,\\
&\leq B(u,u)-\int_U\sum_{i=1}^nb^iu_{x_i}udx-\int_Ucu^2dx,\\
& \leq B(u,u)+\sum_{i=1}^n\Vert b^i\Vert_{L^\infty(U)}\Vert Du\Vert_{L^2(U)}\Vert u\Vert_{L^2(U)}+\Vert c\Vert_{L^\infty(U)}\Vert u\Vert_{L^2(U)}^2.
\end{align*}
By AM-GM inequality, we obtain for any $x,y\leq 0$ and $\epsilon>0$, we have,
\begin{equation*}
xy\leq \varepsilon x^2+{\frac 1 {4\varepsilon}}y^2.
\end{equation*}
Let us choose $\varepsilon>0$ to be such that
\begin{equation*}
\varepsilon = {\frac \theta {2\sum_{i=1}^n\Vert b^i\Vert_{L^\infty(U)}}}.
\end{equation*}
Substituting this to the inequality, we get
\begin{align*}
\theta\int_U\vert Du\vert^2&\leq B(u,u)+{\frac \theta 2}\Vert u\Vert^2_{H^1_0(U)}+\gamma\Vert u\Vert_{L^2(U)}^2,\\
{\frac 1 2}\theta\Vert u\Vert_{H^1_0(U)}^2 &\leq B(u,u)+\gamma\Vert u\Vert^2_{L^2(U)}.
\end{align*}
Let $\beta={\frac 1 2}$, we derived the claim.
\end{proof}

\begin{remark}
In general the problem
\begin{equation*}
Lu=f,\quad u\in H_0^1(U),
\end{equation*}
is not solvable. 
\end{remark}

\begin{theorem}
Let the second order elliptic operator $L$ with boundaries. There exists $\gamma\geq 0$ such that for any $\mu\geq\gamma$ and each $f\in\ L^2(U)$, there exists a unique solution $u\in H_0^1(U)$ such that
\begin{equation*}
B(u,v)+\mu\langle u,v\rangle_{H_0^1(U)}=\langle f, v\rangle_{H_0^1(U)}
\end{equation*}
holds for any $v\in H_0^1(U)$. 
\end{theorem}

\begin{proof}
Let $\gamma\geq 0$ be as in Theorem \ref{LM_modified}. Now we define
\begin{equation*}
B_\mu(u,v) = B(u,v) +\mu\langle u,v\rangle_{L^2(U)}.
\end{equation*}
Then we have
\begin{align*}
B_\mu(u,v)&\leq \vert B(u,v)\vert+\mu\vert\langle u,v\rangle_{L^2(U)}\vert,\\
&\leq\alpha\Vert u \Vert_{H_0^1(U)}\Vert v\Vert_{H_0^1(U)}+\mu \Vert u\Vert_{L^2(U)}\Vert v\Vert_{L^2(U)},\\
& \leq (\alpha+\mu)\Vert u \Vert_{H_0^1(U)}\Vert v\Vert_{H_0^1(U)}.
\end{align*}
Thus satisfies the first condition for Theorem \ref{LM}. For the second part, we observe that
\begin{align*}
\beta\Vert u \Vert_{H_0^1(U)}^2&\leq B(u,v)+\gamma\Vert u \Vert_{H_0^1(U)}\Vert v\Vert_{H_0^1(U)}^2,\\
& B_\mu(u,u)-\mu\Vert u\Vert_{L^2(U)}^2+\gamma\Vert u \Vert_{H_0^1(U)}\Vert v\Vert_{H_0^1(U)}^2,\\
&\leq B_\mu(u,u).
\end{align*}
Therefore, there is a unique $u\in H_0^1(U)$, such that for any $v\in H_0^1(U)$, we have
\begin{equation*}
B_\mu(u,v) = \langle f,v\rangle_{L^2(U)}.
\end{equation*}
(In other words, the problem has a weak solution).
\end{proof}

\begin{remark}

\end{remark}

\end{document}